% Monograph LaTeX Template for UFSC based on:
%
% 1. https://github.com/royertiago/tcc
% 2. http://portal.bu.ufsc.br/normalizacao/
% 3. https://github.com/AdrianoRuseler/abntex2-ufsc

% You need to run `pdfTeX` 5 times on the following order: 1. `pdfTeX`, 2. `biber`, 3. `pdfTeX` 4.
% `pdfTeX` 5. `pdfTeX` 6. `biber` 7. `pdfTeX`, when the bibliography includes a cyclic reference to
% another bibliography, so we need a last pass to fix the bibliography undefined references.

% Includes and fixes several `abntex2` class problems
% Must be included before loading class
\input{setup/setup}

% Uses abntex2 class
\documentclass[
10pt, %Padrão UFSC para versão final
a5paper, %Padrão UFSC para versão final
% 12pt, %Pode usar tamanho 12pt para defesa
% a4paper, %Pode usar a4 para defesa
twoside, % Impressão nos dois lados da folha
chapter=TITLE, % Título de capítulos em caixa alta
section=TITLE  % Título de seções em caixa alta
]{abntex2}
% The UFSC font size is 10.5, but memoir embbeded on `abntex2` only accepts 10 and 11pt. However
% this will be fixed later by the ufscthesisx package.
% You can see whether the font sizes are correct with the \f@size at the `\begin{document}`.


%----------------------------------------------------------------------------------------
%   MINOR CORRECTIONS BEFORE LOADING UFSCThesis settings
%----------------------------------------------------------------------------------------
% How can I put more space between bibliography entries (biblatex)
% https://tex.stackexchange.com/questions/19105/how-can-i-put-more-space-between-bibliography-entries-biblatex
% \setlength\bibitemsep{2.1\itemsep}


% % Reduce font size of bibliography; overfull bibliography
% % https://tex.stackexchange.com/questions/203764/reduce-font-size-of-bibliography-overfull-bibliography
% \newcommand{\bibliographyfontsize}{\fontsize{10.0pt}{10.5pt}\selectfont}
% \renewcommand*{\bibfont}{\bibliographyfontsize}

% % How to correctly insert and justify abstract paragraph on my bibliography?
% % https://tex.stackexchange.com/questions/398666/how-to-correctly-insert-and-justify-abstract
% \DeclareFieldFormat{abstract}%
% {%
%     \vspace*{-0.5mm}\par\justifying
%     \begin{adjustwidth}{1cm}{}
%         \textbf{\bibsentence\bibstring{abstract}:} #1
%     \end{adjustwidth}
% }

% \renewbibmacro*{finentry}%
% {%
%     \iffieldundef{abstract}
%     {\finentry}
%     {\finentrypunct
%         \printfield{abstract}%
%         \renewcommand*{\finentrypunct}{}%
%         \finentry
%     }
% }

%----------------------------------------------------------------------------------------
%   LOAD UFSCThesis style
%----------------------------------------------------------------------------------------
\usepackage{setup/ufscthesisx}

% Load all required basic packages
\input{setup/utilities}




% FIXME Substituir 'Nome completo do autor' pelo seu nome.
\autor{Nome completo do autor(a)}

% FIXME Substituir 'Título do trabalho' pelo título da trabalho.
\titulo{Título do trabalho}
% FIXME Substituir 'Subtítulo (se houver)' pelo subtítulo da trabalho.  
% Caso não tenha substítulo, comente a linha a seguir.
\subtitulo{Subtítulo (se houver)}
% FIXME Se estiver defendendo uma dissertação de mestrado, descomente a linha a seguir.
% \def\mestrado{}



% FIXME Substituir 'Nome do orientador' pelo nome do seu
% orientador.
% \orientador{Nome do orientador, Dr.}
% FIXME Se for orientado por uma mulher, descomente a linha acima e comente a linha a seguir.
\orientador[Orientadora]{Nome da orientadora, Dra.}


% FIXME Substituir 'Nome do coorientador' pelo nome do seu
% coorientador. Caso não tenha coorientador, comente a linha a seguir.
\coorientador{Nome do coorientador, Dr.}
% FIXME Se for coorientado por uma mulher, comente a linha acima e descomente a linha a seguir.
% \coorientador[Coorientadora]{Nome da coorientadora, Dra.}


% FIXME Se for coorientado por uma mulher, descomente a linha a seguir.

% FIXME Substituir 'Nome do Coordenador' pelo nome do Coordenador do 
% programa/curso.
\coordenador{Nome do Coordenador, Dr.}
% FIXME Se for coordenadora mulher, comente a linha acima e descomente a linha a seguir.
% \coordenador[Coordenadora]{Nome da Coordenadora, Dra.}

% FIXME Substituir '2018' pelo ano (ou mês e ano) em que seu trabalho foi defendido.
\ano{2018}

% FIXME Substituir '\today' pela data em que ocorreu sua defesa.
\data{\today}

% FIXME Substituir 'Local' pela cidade em que ocorreu sua defesa.
\local{Local}

\instituicaosigla{UFSC}
\instituicao{Universidade Federal de Santa Catarina}
\biblioteca{Biblioteca Universitária}


% FIXME Substituir 'Tese' pelo tipo de trabalho (Tese, Dissertação, Monografia, Trabalho de Conclusão de Curso). 
\tipotrabalho{Tese}
% FIXME Se estiver apresentando um Trabalho de Conclusão de Curso, descomente a linha a seguir.
% \def\tcc{}

% FIXME Substituir 'Área de Concentração' pela área de concentração. 
% Caso não tenha área de concentração, comente a linha a seguir.
% \area{Área de Concentração}


% FIXME Substituir 'Doutor em XXXXXX' pela grau adequado.
\formacao{Doutor em XXXXXX}

% FIXME Substituir 'Programa de Pós-Graduação em XXXXXX' pela curso adequado.
\programa{Programa de Pós-Graduação em XXXXXX}

% FIXME Substituir 'Departamento de XXXXXX, Centro de XXXXXX' pela departamento e centro adequado.
\centro{Departamento de XXXXXX, Centro de XXXXXX}


%----------------------------------------------------------------------------------------
%   Inserir palavras-chave aqui para aparecer na ficha-catalográfica.
%----------------------------------------------------------------------------------------

% Coloque aqui as palavras-chave em portugues
\newcommand{\palavraschaveportugues}
{%
    \item \LaTeX. \item \abnTeX. \item editoração de texto.
}
% Insert here the keyworkds in english
\newcommand{\palavraschaveingles}
{%
    \item \LaTeX. \item \abnTeX. \item text publishing.
}


% Altere o arquivo settings.tex para incluir suas configurações.
%----------------------------------------------------------------------------------------
%   File settings
%----------------------------------------------------------------------------------------

% Comment this, unless you are debugging pages' badness Underfull & Overflow
% https://tex.stackexchange.com/questions/115908/geometry-showframe-landscape
%
% What is the difference between \usepackage{showframe} and \usepackage[showframe]{geometry}?
% https://tex.stackexchange.com/questions/387077/what-is-the-difference-between-usepackageshowframe-and-usepackageshowframe
%
% How to do the memoir headings fix but not have my text going over the page bottom margins?
% https://tex.stackexchange.com/questions/387257/how-to-do-the-memoir-headings-fix-but-not-have-my-text-going-over-the-page-botto
%
% Print page margins of a document, for debugging bad boxes
% https://tex.stackexchange.com/questions/14508/print-page-margins-of-a-document
% \usepackage[showframe,pass]{geometry}


% To use the font Times New Roman, instead of the default LaTeX font
% more up-to-date than '\usepackage{mathptmx}'
% \usepackage{newtxtext}
% \usepackage{newtxmath}

% Novo list of (listings) para QUADROS usando newfloat
\usepackage{newfloat}
\makeatletter
%% we define a helper macro for adjusting lists of new floats to
%% accept a * behind them for not being shown in the TOC, like
%% the other list printing commands in memoir
% https://tex.stackexchange.com/questions/176418/remove-list-of-newfloat-from-toc
\newcommand{\AdjustForMemoir}[1]{%
  \csletcs{kept@listof#1}{listof#1}%
  \csdef{listof#1}{%
    \@ifstar
     {\csappto{newfloat@listof#1@hook}{\append@star}%
      \csuse{kept@listof#1}}%
     {\csuse{kept@listof#1}}%
  }
}
\def\append@star#1{#1*}
\makeatother
% \newcommand{\quadroname}{Quadro}
% \newcommand{\listofquadrosname}{Lista de Quadros}
\DeclareFloatingEnvironment[fileext=loq,placement={!hbtp},name=Quadro,within=chapter,listname=Lista de Quadros]{quadro}
\AdjustForMemoir{quadro}
\newlistentry{quadro}{loq}{0}



% Thesis settings
\newcommand{\brazilword}[1]{\foreignlanguage{brazil}{#1}}
\newcommand{\englishword}[1]{\foreignlanguage{english}{#1}}

\lang % Switch between english and brazil for big text blocks
{\includecomment{englishtext}\excludecomment{braziltext}}
{\includecomment{braziltext}\excludecomment{englishtext}}

% What is the difference between \def and \newcommand?
% https://tex.stackexchange.com/questions/655/what-is-the-difference-between-def-and-newcommand
\def\mytextpreliminarylistname{\lang{Brief Table of Contents}{Breve Sumário}}

% How to manually set where a word is split?
% https://tex.stackexchange.com/questions/182569/how-to-manually-set-where-a-word-is-split
\hyphenation{Ge-la-im}


% Patch the `abntex2` citacao environment
\xpatchcmd{\citacao}
{\list{}}
{\list{}{\topsep=0pt}}
{}
{\PackageWarning{ufscthesisx}{Citation \topsep is not patched. The 'citacao' environment must be
patched with 'topsep=0pt' but it failed. This is probably due an update on the main class 'abnTeX2',
therefore the \topsep patch must need to be updated accordingly}}

% Remove the colon appended to theses variables, allowing us to use other separators
\addto\captionsbrazil
{
    \renewcommand{\orientadorname}{Orientador}
    \renewcommand{\coorientadorname}{Coorientador}
}

% Create caption English translations as the sections headers
% https://tex.stackexchange.com/questions/8564/what-is-the-right-way-to-redefine-macros-defined-by-babel
\addto\captionsenglish
{
    %% adjusts names from abnTeX2
    \renewcommand{\folhaderostoname}{Title page}
    \renewcommand{\epigraphname}{Epigraph}
    \renewcommand{\dedicatorianame}{Dedication}
    \renewcommand{\errataname}{Errata sheet}
    \renewcommand{\agradecimentosname}{Acknowledgements}
    \renewcommand{\anexoname}{ANNEX}
    \renewcommand{\anexosname}{Annex}
    \renewcommand{\apendicename}{APPENDIX}
    \renewcommand{\apendicesname}{Appendix}
    \renewcommand{\orientadorname}{Supervisor}
    \renewcommand{\coorientadorname}{Co\hyp{}supervisor}
    \renewcommand{\folhadeaprovacaoname}{Approval}
    \renewcommand{\resumoname}{Abstract}
    \renewcommand{\listadesiglasname}{List of abbreviations and acronyms}
    \renewcommand{\listadesimbolosname}{List of symbols}
    \renewcommand{\fontename}{Source}
    \renewcommand{\notaname}{Note}
    %% adjusts names used by \autoref
    \renewcommand{\pageautorefname}{page}
    \renewcommand{\sectionautorefname}{section}
    \renewcommand{\subsectionautorefname}{subsection}
    \renewcommand{\subsubsectionautorefname}{subsubsection}
    \renewcommand{\paragraphautorefname}{subsubsubsection}
}

% Source Code Settings in Document
\makeatletter
\@ifpackageloaded{listings}
{
\ifenglish
    % These default values are already in English
\else
    % Listing -> Codigo fonte
    \renewcommand{\lstlistingname}{Código--fonte}

    % List of Listings -> Lista de códigos-fonte
    \renewcommand{\lstlistlistingname}{Lista de códigos--fonte}

    % Calculate the size of the header
    \calculatelisteningsheader
\fi
}{}
\makeatother


% Espaçamentos entre linhas e parágrafos
%
% ifpackageloaded question
% https://tex.stackexchange.com/questions/70212/ifpackageloaded-question
\makeatletter
\@ifclassloaded{memoir}
{
    % Estilo de capítulos, ver classe para maiores detalhes.Veja outros estilos em:
    % http://mirrors.ibiblio.org/CTAN/macros/latex/contrib/memoir/memman.pdf
    \chapterstyle{VZ14}
    \setlength\beforechapskip{0pt}
    \setlength\midchapskip{15pt}
    \setlength\afterchapskip{15pt}

    % O tamanho do parágrafo é dado por:
    \setlength{\parindent}{1.3cm}

    % Controle do espaçamento entre um parágrafo e outro. Tente também
    % \onelineskip
    \setlength{\parskip}{0.2cm}

    % Memoir: Warnings “The material used in the headers is too large” w/ accented titles
    % https://tex.stackexchange.com/questions/387293/how-to-change-the-page-layout-with-memoir
    \setheadfoot{30.0pt}{\footskip}
    \checkandfixthelayout
}{}
\makeatother


% Color settings across the document
\makeatletter
\@ifpackageloaded{xcolor}
{
    % RGB colors in absolute values from 0 to 255 by using `RGB` tag
    \definecolor{darkblue}{RGB}{26,13,178}

    % Definição de cores, RGB colors in percentage notation by using `rgb` tag
    \definecolor{mygreen}{rgb}{0,0.6,0}
    \definecolor{mygray}{rgb}{0.5,0.5,0.5}
    \definecolor{mymauve}{rgb}{0.58,0,0.82}

    % Configurações de aparência do PDF final
    \definecolor{figcolor}{rgb}{1,0.4,0}  % orange
    \definecolor{tabcolor}{rgb}{1,0.4,0}  % orange
    \definecolor{eqncolor}{rgb}{1,0.4,0}  % orange
    \definecolor{linkcolor}{rgb}{1,0.4,0} % orange
    \definecolor{citecolor}{rgb}{1,0.4,0} % orange
    \definecolor{seccolor}{rgb}{0,0,1}    % blue
    \definecolor{abscolor}{rgb}{0,0,1}    % blue
    \definecolor{titlecolor}{rgb}{0,0,1}  % blue
    \definecolor{biocolor}{rgb}{0,0,1}    % blue

    % Alterando o aspecto da cor azul
    \definecolor{blue}{RGB}{41,5,195}

    % Informações do PDF
    \@ifpackageloaded{hyperref}
    {
        \hypersetup
        {
            pdftitle={\@title},
            colorlinks=true, % false: boxed links; true: colored links
            linkcolor=darkblue, % color of internal links
            citecolor=darkgreen, % color of links to bibliography
            filecolor=black, % color of file links
            urlcolor=linkcolor,
            bookmarksdepth=4
        }
        \ifenglish
            \hypersetup
            {
                pdfauthor={Author},
                pdfsubject={Thesis' Abstract},
                pdfcreator={LaTeX with abnTeX2 for UFSC},
                pdfkeywords={abnt}{latex}{UFSC}{abntex2}{thesis},
            }
        \else
            \hypersetup
            {
                pdfauthor={Autores},
                pdfsubject={Resumo da tese},
                pdfcreator={LaTeX com abnTeX2 para UFSC},
                pdfkeywords={abnt}{latex}{UFSC}{abntex2}{tese},
            }
        \fi
    }
}{}
\makeatother


% Changing the font of the numbers in the ToC in the memoir class
% https://tex.stackexchange.com/questions/14314/changing-the-font-of-the-numbers-in-the-toc-in-the-memoir-class
\renewcommand{\cftpartfont}{\ABNTEXpartfont\color{ultramarine}}
\renewcommand{\cftpartpagefont}{\ABNTEXpartfont\color{black}}

\renewcommand{\cftchapterfont}{\ABNTEXchapterfont\color{ultramarine}}
\renewcommand{\cftchapterpagefont}{\ABNTEXchapterfont\color{black}}

\renewcommand{\cftsectionfont}{\ABNTEXsectionfont\color{ultramarine}}
\renewcommand{\cftsectionpagefont}{\ABNTEXsectionfont\color{black}}

\renewcommand{\cftsubsectionfont}{\ABNTEXsubsectionfont\color{ultramarine}}
\renewcommand{\cftsubsectionpagefont}{\ABNTEXsubsectionfont\color{black}}

\renewcommand{\cftsubsubsectionfont}{\ABNTEXsubsubsectionfont\color{ultramarine}}
\renewcommand{\cftsubsubsectionpagefont}{\ABNTEXsubsubsectionfont\color{black}}

\renewcommand{\cftparagraphfont}{\ABNTEXsubsubsubsectionfont\color{ultramarine}}
\renewcommand{\cftparagraphpagefont}{\ABNTEXsubsubsubsectionfont\color{black}}


% Backref package settings, pages with citations in bibliography
\makeatletter
\@ifpackageloaded{biblatex}
{
    \renewbibmacro*{pageref}
    {
        \iflistundef{pageref}
        {\printtext{\lang{\autocap{n}o citation in the text}{\autocap{n}enhuma citação no texto}.}}
        {%
            \printtext
            {%
                \ifnumgreater{\value{citecounter}}{1}
                    {\lang{\autocap{c}ited}{\autocap{c}itado} \arabic{citecounter} \lang{times}{vezes}}
                    {\lang{\autocap{c}ited}{\autocap{c}itado} \arabic{citecounter} \lang{time}{vez}}
            }%
            \setunit{\addspace}%
            \ifnumgreater{\value{pageref}}{1}
                {\bibstring{backrefpages}\ppspace}
                {\bibstring{backrefpage}\ppspace}%
            \printlist[pageref][-\value{listtotal}]{pageref}%
        }%
    }

    \DefineBibliographyStrings{brazil}
    {
        backrefpage  = {na página},
        backrefpages = {nas páginas},
    }

    \DefineBibliographyStrings{english}
    {
        backrefpage  = {on page},
        backrefpages = {on pages},
    }
}{}
\makeatother


% Memoir has another mechanism for the job: \cftsetindents{‹kind›}{indent}{numwidth}. Here kind is
% chapter, section, or whatever; the indent specifies the ‘margin’ before the entry starts; and the
% width is of the box into which the number is typeset (so needs to be wide enough for the largest
% number, with the necessary spacing to separate it from what comes after it in the line.
% http://www.tex.ac.uk/FAQ-tocloftwrong.html
%
% memoir: indentation of unnumbered sections in table of contents
% https://tex.stackexchange.com/questions/264668/memoir-indentation-of-unnumbered-sections-in-table-of-contents
%
% Memoir ToC: indent the second line by number+space width in the previous line OR set standard
% https://tex.stackexchange.com/questions/394227/memoir-toc-indent-the-second-line-by-numberspace

% `\cftlastnumwidth` and these `\cftsetindents` are defined by the abntex2 class saying they obey
% the `ABNTEXsumario-abnt-6027-2012`.
% \newlength{\cftlastnumwidth}

\setlength{\cftlastnumwidth}{\cftsubsubsectionnumwidth}
\addtolength{\cftlastnumwidth}{-1em}

% http://www.tex.ac.uk/FAQ-tocloftwrong.html
% Use \setlength\cftsectionnumwidth{4em} to override all these values
\cftsetindents{part}         {0em}{\cftlastnumwidth}
\cftsetindents{chapter}      {0em}{\cftlastnumwidth}
\cftsetindents{section}      {0em}{\cftlastnumwidth}
\cftsetindents{subsection}   {0em}{\cftlastnumwidth}
\cftsetindents{subsubsection}{0em}{\cftlastnumwidth}
\cftsetindents{paragraph}    {0em}{\cftlastnumwidth}
\cftsetindents{subparagraph} {0em}{\cftlastnumwidth}





% Utilize o arquivo aftertext/references.bib para incluir sua bibliografia.
\addbibresource{aftertext/references.bib}


% When writing a large document, it is sometimes useful to work on selected sections of the document
% to speed up compilation time: https://en.wikibooks.org/wiki/TeX/includeonly
\newif\ifforcedinclude\forcedincludefalse

% \addtoincludeonly{beforetext/agradecimentos}
% \addtoincludeonly{beforetext/epigrafe}
% \addtoincludeonly{beforetext/fichacatalografica}
% \addtoincludeonly{beforetext/folhadeaprovacao}
% \addtoincludeonly{beforetext/resumos}
% \addtoincludeonly{beforetext/siglas}
% \addtoincludeonly{beforetext/simbolos}

% \addtoincludeonly{chapters/intro}
% \addtoincludeonly{chapters/chapter_1}
% \addtoincludeonly{chapters/chapter_2}
% \addtoincludeonly{chapters/chapter_3}
% \addtoincludeonly{chapters/chapter_4}
% \addtoincludeonly{chapters/conclusion}

% \addtoincludeonly{aftertext/anexo_a}
% \addtoincludeonly{aftertext/anexo_b}
% \addtoincludeonly{aftertext/apendice_a}

% Control whether the full document will be generated
% \doincludeonly
% \forcedincludetrue



% %----------------------------------------------------------------------------------------
% %   DOCUMENT CONTENTS
% %----------------------------------------------------------------------------------------
\begin{document}

    % Comment this after finishing the thesis, so you can start fixing the \flushbottom vs \raggedbottom
    % https://tex.stackexchange.com/questions/65355/flushbottom-vs-raggedbottom
    \raggedbottom

    % Retira espaço extra obsoleto entre as frases `Double space between sentences`
    % https://tex.stackexchange.com/questions/4705/double-space-between-sentences
    \frenchspacing

    % ELEMENTOS PRÉ-TEXTUAIS
    \input{setup/beforetext}



    % ELEMENTOS TEXTUAIS
    \textual

    % To automatically put a [Go To Top/Back] ←← | ← on each section
    % \addGoToSummary

    \setlength\beforechapskip{50pt}
    \setlength\midchapskip{20pt}
    \setlength\afterchapskip{20pt}

    % Introdução (exemplo de capítulo sem numeração, mas presente no Sumário)
    \include{chapters/intro}

    % PARTE
    % \ifforcedinclude\else\part{Preparação da pesquisa}\fi

    % Capitulo com exemplos de comandos inseridos de arquivo externo
    \include{chapters/chapter_1}

    % PARTE
    % \ifforcedinclude\else\part{Referenciais teóricos}\fi

    % Capitulo de revisão de literatura
    \include{chapters/chapter_2}

    % PARTE
    % \ifforcedinclude\else\part{Resultados}\fi

    % Primeiro capitulo de Resultados
    
% The \phantomsection command is needed to create a link to a place in the document that is not a
% figure, equation, table, section, subsection, chapter, etc.
\cleardoublepage
\phantomsection


% \MakeUppercase in \section and \chapter with hyperref cause trouble
% https://tex.stackexchange.com/questions/199374/makeuppercase-in-section-and-chapter-with-hyperref-cause-trouble
\addcontentsline{toc}{chapter}{\HyperrefUppercase\chooselang{Chapter 3}{Capítulo 3}}


\chapter{Lorem ipsum dolor sit amet}

\begin{flushright}
    \showfont
\end{flushright}

\newpage


% ---
\section{Aliquam vestibulum fringilla lorem}
% ---

Lipsum me [1]

Lipsum me [2-3]


    % Segundo capitulo de Resultados
    
% The \phantomsection command is needed to create a link to a place in the document that is not a
% figure, equation, table, section, subsection, chapter, etc.
\cleardoublepage
\phantomsection


% \MakeUppercase in \section and \chapter with hyperref cause trouble
% https://tex.stackexchange.com/questions/199374/makeuppercase-in-section-and-chapter-with-hyperref-cause-trouble
\addcontentsline{toc}{chapter}{\HyperrefUppercase{Capítulo 4}}


\chapter{Lectus lobortis condimentum}
% ---

\begin{flushright}
    \showfont
\end{flushright}

\newpage

% ---
\section{Vestibulum ante ipsum primis in faucibus orci luctus et ultrices
    posuere cubilia Curae}
% ---

\lipsum[21-22]

\showfont




    % Finaliza a parte no bookmark do PDF
    % para que se inicie o bookmark na raiz
    % e adiciona espaço de parte no Sumário
    \phantompart

    % Conclusão (outro exemplo de capítulo sem numeração e presente no sumário)
    

% The \phantomsection command is needed to create a link to a place in the document that is not a
% figure, equation, table, section, subsection, chapter, etc.
%
% When do I need to invoke \phantomsection?
% https://tex.stackexchange.com/questions/44088/when-do-i-need-to-invoke-phantomsection
\phantomsection

% ---
% Considerações Finais (outro exemplo de capítulo sem numeração e presente no sumário)
% ---
\chapter*[]{\lang{Final Remarks}{Considerações Finais}}
\phantomsection
\addcontentsline{toc}{chapter}{Considerações Finais}%

    Lipsum me [31-33]






    % ELEMENTOS PÓS-TEXTUAIS
    \postextual
    \setlength\beforechapskip{0pt}
    \setlength\midchapskip{15pt}
    \setlength\afterchapskip{15pt}

    % Referências bibliográficas
    \begingroup
        % Using BibTeX to make a list of references without having citations in the body of the document?
        % https://tex.stackexchange.com/questions/17128/using-bibtex-to-make-a-list-of-references-without
        % \nocite{*}

        % How to modify line spacing per entry of bibliography?
        % https://tex.stackexchange.com/questions/163559/how-to-modify-line-spacing-per-entry-of-bibliography
        \linespread{1.18}\selectfont
        \printbibliography
    \endgroup

    % Glossário, consulte o manual da classe abntex2 para orientações sobre o glossário.
    % \ifforcedinclude\else\glossary\fi

    % Apêndices, inicia os apêndices
    \begin{apendicesenv}
        % Imprime uma página indicando o início dos apêndices
        \ifforcedinclude\else\partapendices\fi

        \setlength\beforechapskip{50pt}
        \setlength\midchapskip{20pt}
        \setlength\afterchapskip{20pt}

        \include{aftertext/apendice_a}
    \end{apendicesenv}

    % Anexos, inicia os anexos
    \begin{anexosenv}
        % Imprime uma página indicando o início dos anexos
        \ifforcedinclude\else\partanexos\fi

        \setlength\beforechapskip{50pt}
        \setlength\midchapskip{20pt}
        \setlength\afterchapskip{20pt}

        \include{aftertext/anexo_a}
        \include{aftertext/anexo_b}
    \end{anexosenv}

    % INDICE REMISSIVO
    \ifforcedinclude\else
        \phantompart
        \printindex
    \fi

\end{document}

