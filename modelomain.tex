
%
% Monograph LaTeX Template based on:
%
% 1. https://github.com/royertiago/tcc
% 2. http://portal.bu.ufsc.br/normalizacao/
% 3. https://github.com/AdrianoRuseler/abntex2-ufsc

% You need to run `pdfTeX` 5 times on the following order: 1. `pdfTeX`, 2. `biber`, 3. `pdfTeX` 4.
% `pdfTeX` 5. `pdfTeX` 6. `biber` 7. `pdfTeX`, when the bibliography includes a cyclic reference to
% another bibliography, so we need a last pass to fix the bibliography undefined references.

%----------------------------------------------------------------------------------------
%   PACKAGES AND OTHER DOCUMENT CONFIGURATIONS
%----------------------------------------------------------------------------------------

% Uncomment the line `\englishtrue` to set the document default language to english.
%
% Is it possible to keep my translation together with original text?
% https://tex.stackexchange.com/questions/5076/is-it-possible-to-keep-my-translation-together-with-original-text
\newif\ifenglish\englishfalse
% \englishtrue

% How to expand \ifthenelse so that it can be used in \parshape?
% https://tex.stackexchange.com/questions/131002/how-to-expand-ifthenelse-so-that-it-can-be-used-in-parshape
\newcommand{\chooselang}[2]{\ifenglish#1\else#2\fi}

% How to make \PassOptionsToPackage add the option as the last option?
% https://tex.stackexchange.com/questions/385895/how-to-make-passoptionstopackage-add-the-option-as-the-last
\ifenglish
    \newcommand{\swapcontents}[2]{#1 #2}

    \PassOptionsToPackage{language=english}{biblatex}
    \PassOptionsToPackage{brazil,main=english,spanish,french}{babel}
\else
    \newcommand{\swapcontents}[2]{#2 #1}

    \PassOptionsToPackage{language=brazil}{biblatex}
    \PassOptionsToPackage{main=brazil,english,spanish,french}{babel}
\fi


% Includes and fixes several `abntex2` class problems
\input{setup/setup}

% Load the UFSC thesis settings
\usepackage{setup/ufscthesisx}

% Load all required basic packages
\input{setup/utilities}



%----------------------------------------------------------------------------------------
%   File settings
%----------------------------------------------------------------------------------------

% Comment this, unless you are debugging pages' badness Underfull & Overflow
% https://tex.stackexchange.com/questions/115908/geometry-showframe-landscape
%
% What is the difference between \usepackage{showframe} and \usepackage[showframe]{geometry}?
% https://tex.stackexchange.com/questions/387077/what-is-the-difference-between-usepackageshowframe-and-usepackageshowframe
%
% How to do the memoir headings fix but not have my text going over the page bottom margins?
% https://tex.stackexchange.com/questions/387257/how-to-do-the-memoir-headings-fix-but-not-have-my-text-going-over-the-page-botto
%
% Print page margins of a document, for debugging bad boxes
% https://tex.stackexchange.com/questions/14508/print-page-margins-of-a-document
% \usepackage[showframe,pass]{geometry}

% To use the font Times New Roman, instead of the default LaTeX font
% more up-to-date than '\usepackage{mathptmx}'
% \usepackage{newtxtext}
% \usepackage{newtxmath}

% Always use it as should improve full justification
% https://tex.stackexchange.com/questions/10377/texttt-overfull-hbox-problem
% https://tex.stackexchange.com/questions/66052/should-i-load-microtype-with-pdflatex
\usepackage{microtype}

% Thesis settings
%----------------------------------------------------------------------------------------
%   File settings
%----------------------------------------------------------------------------------------

% Comment this, unless you are debugging pages' badness Underfull & Overflow
% https://tex.stackexchange.com/questions/115908/geometry-showframe-landscape
%
% What is the difference between \usepackage{showframe} and \usepackage[showframe]{geometry}?
% https://tex.stackexchange.com/questions/387077/what-is-the-difference-between-usepackageshowframe-and-usepackageshowframe
%
% How to do the memoir headings fix but not have my text going over the page bottom margins?
% https://tex.stackexchange.com/questions/387257/how-to-do-the-memoir-headings-fix-but-not-have-my-text-going-over-the-page-botto
%
% Print page margins of a document, for debugging bad boxes
% https://tex.stackexchange.com/questions/14508/print-page-margins-of-a-document
% \usepackage[showframe,pass]{geometry}


% To use the font Times New Roman, instead of the default LaTeX font
% more up-to-date than '\usepackage{mathptmx}'
% \usepackage{newtxtext}
% \usepackage{newtxmath}

% Novo list of (listings) para QUADROS usando newfloat
\usepackage{newfloat}
\makeatletter
%% we define a helper macro for adjusting lists of new floats to
%% accept a * behind them for not being shown in the TOC, like
%% the other list printing commands in memoir
% https://tex.stackexchange.com/questions/176418/remove-list-of-newfloat-from-toc
\newcommand{\AdjustForMemoir}[1]{%
  \csletcs{kept@listof#1}{listof#1}%
  \csdef{listof#1}{%
    \@ifstar
     {\csappto{newfloat@listof#1@hook}{\append@star}%
      \csuse{kept@listof#1}}%
     {\csuse{kept@listof#1}}%
  }
}
\def\append@star#1{#1*}
\makeatother
% \newcommand{\quadroname}{Quadro}
% \newcommand{\listofquadrosname}{Lista de Quadros}
\DeclareFloatingEnvironment[fileext=loq,placement={!hbtp},name=Quadro,within=chapter,listname=Lista de Quadros]{quadro}
\AdjustForMemoir{quadro}
\newlistentry{quadro}{loq}{0}



% Thesis settings
\newcommand{\brazilword}[1]{\foreignlanguage{brazil}{#1}}
\newcommand{\englishword}[1]{\foreignlanguage{english}{#1}}

\lang % Switch between english and brazil for big text blocks
{\includecomment{englishtext}\excludecomment{braziltext}}
{\includecomment{braziltext}\excludecomment{englishtext}}

% What is the difference between \def and \newcommand?
% https://tex.stackexchange.com/questions/655/what-is-the-difference-between-def-and-newcommand
\def\mytextpreliminarylistname{\lang{Brief Table of Contents}{Breve Sumário}}

% How to manually set where a word is split?
% https://tex.stackexchange.com/questions/182569/how-to-manually-set-where-a-word-is-split
\hyphenation{Ge-la-im}


% Patch the `abntex2` citacao environment
\xpatchcmd{\citacao}
{\list{}}
{\list{}{\topsep=0pt}}
{}
{\PackageWarning{ufscthesisx}{Citation \topsep is not patched. The 'citacao' environment must be
patched with 'topsep=0pt' but it failed. This is probably due an update on the main class 'abnTeX2',
therefore the \topsep patch must need to be updated accordingly}}

% Remove the colon appended to theses variables, allowing us to use other separators
\addto\captionsbrazil
{
    \renewcommand{\orientadorname}{Orientador}
    \renewcommand{\coorientadorname}{Coorientador}
}

% Create caption English translations as the sections headers
% https://tex.stackexchange.com/questions/8564/what-is-the-right-way-to-redefine-macros-defined-by-babel
\addto\captionsenglish
{
    %% adjusts names from abnTeX2
    \renewcommand{\folhaderostoname}{Title page}
    \renewcommand{\epigraphname}{Epigraph}
    \renewcommand{\dedicatorianame}{Dedication}
    \renewcommand{\errataname}{Errata sheet}
    \renewcommand{\agradecimentosname}{Acknowledgements}
    \renewcommand{\anexoname}{ANNEX}
    \renewcommand{\anexosname}{Annex}
    \renewcommand{\apendicename}{APPENDIX}
    \renewcommand{\apendicesname}{Appendix}
    \renewcommand{\orientadorname}{Supervisor}
    \renewcommand{\coorientadorname}{Co\hyp{}supervisor}
    \renewcommand{\folhadeaprovacaoname}{Approval}
    \renewcommand{\resumoname}{Abstract}
    \renewcommand{\listadesiglasname}{List of abbreviations and acronyms}
    \renewcommand{\listadesimbolosname}{List of symbols}
    \renewcommand{\fontename}{Source}
    \renewcommand{\notaname}{Note}
    %% adjusts names used by \autoref
    \renewcommand{\pageautorefname}{page}
    \renewcommand{\sectionautorefname}{section}
    \renewcommand{\subsectionautorefname}{subsection}
    \renewcommand{\subsubsectionautorefname}{subsubsection}
    \renewcommand{\paragraphautorefname}{subsubsubsection}
}

% Source Code Settings in Document
\makeatletter
\@ifpackageloaded{listings}
{
\ifenglish
    % These default values are already in English
\else
    % Listing -> Codigo fonte
    \renewcommand{\lstlistingname}{Código--fonte}

    % List of Listings -> Lista de códigos-fonte
    \renewcommand{\lstlistlistingname}{Lista de códigos--fonte}

    % Calculate the size of the header
    \calculatelisteningsheader
\fi
}{}
\makeatother


% Espaçamentos entre linhas e parágrafos
%
% ifpackageloaded question
% https://tex.stackexchange.com/questions/70212/ifpackageloaded-question
\makeatletter
\@ifclassloaded{memoir}
{
    % Estilo de capítulos, ver classe para maiores detalhes.Veja outros estilos em:
    % http://mirrors.ibiblio.org/CTAN/macros/latex/contrib/memoir/memman.pdf
    \chapterstyle{VZ14}
    \setlength\beforechapskip{0pt}
    \setlength\midchapskip{15pt}
    \setlength\afterchapskip{15pt}

    % O tamanho do parágrafo é dado por:
    \setlength{\parindent}{1.3cm}

    % Controle do espaçamento entre um parágrafo e outro. Tente também
    % \onelineskip
    \setlength{\parskip}{0.2cm}

    % Memoir: Warnings “The material used in the headers is too large” w/ accented titles
    % https://tex.stackexchange.com/questions/387293/how-to-change-the-page-layout-with-memoir
    \setheadfoot{30.0pt}{\footskip}
    \checkandfixthelayout
}{}
\makeatother


% Color settings across the document
\makeatletter
\@ifpackageloaded{xcolor}
{
    % RGB colors in absolute values from 0 to 255 by using `RGB` tag
    \definecolor{darkblue}{RGB}{26,13,178}

    % Definição de cores, RGB colors in percentage notation by using `rgb` tag
    \definecolor{mygreen}{rgb}{0,0.6,0}
    \definecolor{mygray}{rgb}{0.5,0.5,0.5}
    \definecolor{mymauve}{rgb}{0.58,0,0.82}

    % Configurações de aparência do PDF final
    \definecolor{figcolor}{rgb}{1,0.4,0}  % orange
    \definecolor{tabcolor}{rgb}{1,0.4,0}  % orange
    \definecolor{eqncolor}{rgb}{1,0.4,0}  % orange
    \definecolor{linkcolor}{rgb}{1,0.4,0} % orange
    \definecolor{citecolor}{rgb}{1,0.4,0} % orange
    \definecolor{seccolor}{rgb}{0,0,1}    % blue
    \definecolor{abscolor}{rgb}{0,0,1}    % blue
    \definecolor{titlecolor}{rgb}{0,0,1}  % blue
    \definecolor{biocolor}{rgb}{0,0,1}    % blue

    % Alterando o aspecto da cor azul
    \definecolor{blue}{RGB}{41,5,195}

    % Informações do PDF
    \@ifpackageloaded{hyperref}
    {
        \hypersetup
        {
            pdftitle={\@title},
            colorlinks=true, % false: boxed links; true: colored links
            linkcolor=darkblue, % color of internal links
            citecolor=darkgreen, % color of links to bibliography
            filecolor=black, % color of file links
            urlcolor=linkcolor,
            bookmarksdepth=4
        }
        \ifenglish
            \hypersetup
            {
                pdfauthor={Author},
                pdfsubject={Thesis' Abstract},
                pdfcreator={LaTeX with abnTeX2 for UFSC},
                pdfkeywords={abnt}{latex}{UFSC}{abntex2}{thesis},
            }
        \else
            \hypersetup
            {
                pdfauthor={Autores},
                pdfsubject={Resumo da tese},
                pdfcreator={LaTeX com abnTeX2 para UFSC},
                pdfkeywords={abnt}{latex}{UFSC}{abntex2}{tese},
            }
        \fi
    }
}{}
\makeatother


% Changing the font of the numbers in the ToC in the memoir class
% https://tex.stackexchange.com/questions/14314/changing-the-font-of-the-numbers-in-the-toc-in-the-memoir-class
\renewcommand{\cftpartfont}{\ABNTEXpartfont\color{ultramarine}}
\renewcommand{\cftpartpagefont}{\ABNTEXpartfont\color{black}}

\renewcommand{\cftchapterfont}{\ABNTEXchapterfont\color{ultramarine}}
\renewcommand{\cftchapterpagefont}{\ABNTEXchapterfont\color{black}}

\renewcommand{\cftsectionfont}{\ABNTEXsectionfont\color{ultramarine}}
\renewcommand{\cftsectionpagefont}{\ABNTEXsectionfont\color{black}}

\renewcommand{\cftsubsectionfont}{\ABNTEXsubsectionfont\color{ultramarine}}
\renewcommand{\cftsubsectionpagefont}{\ABNTEXsubsectionfont\color{black}}

\renewcommand{\cftsubsubsectionfont}{\ABNTEXsubsubsectionfont\color{ultramarine}}
\renewcommand{\cftsubsubsectionpagefont}{\ABNTEXsubsubsectionfont\color{black}}

\renewcommand{\cftparagraphfont}{\ABNTEXsubsubsubsectionfont\color{ultramarine}}
\renewcommand{\cftparagraphpagefont}{\ABNTEXsubsubsubsectionfont\color{black}}


% Backref package settings, pages with citations in bibliography
\makeatletter
\@ifpackageloaded{biblatex}
{
    \renewbibmacro*{pageref}
    {
        \iflistundef{pageref}
        {\printtext{\lang{\autocap{n}o citation in the text}{\autocap{n}enhuma citação no texto}.}}
        {%
            \printtext
            {%
                \ifnumgreater{\value{citecounter}}{1}
                    {\lang{\autocap{c}ited}{\autocap{c}itado} \arabic{citecounter} \lang{times}{vezes}}
                    {\lang{\autocap{c}ited}{\autocap{c}itado} \arabic{citecounter} \lang{time}{vez}}
            }%
            \setunit{\addspace}%
            \ifnumgreater{\value{pageref}}{1}
                {\bibstring{backrefpages}\ppspace}
                {\bibstring{backrefpage}\ppspace}%
            \printlist[pageref][-\value{listtotal}]{pageref}%
        }%
    }

    \DefineBibliographyStrings{brazil}
    {
        backrefpage  = {na página},
        backrefpages = {nas páginas},
    }

    \DefineBibliographyStrings{english}
    {
        backrefpage  = {on page},
        backrefpages = {on pages},
    }
}{}
\makeatother


% Memoir has another mechanism for the job: \cftsetindents{‹kind›}{indent}{numwidth}. Here kind is
% chapter, section, or whatever; the indent specifies the ‘margin’ before the entry starts; and the
% width is of the box into which the number is typeset (so needs to be wide enough for the largest
% number, with the necessary spacing to separate it from what comes after it in the line.
% http://www.tex.ac.uk/FAQ-tocloftwrong.html
%
% memoir: indentation of unnumbered sections in table of contents
% https://tex.stackexchange.com/questions/264668/memoir-indentation-of-unnumbered-sections-in-table-of-contents
%
% Memoir ToC: indent the second line by number+space width in the previous line OR set standard
% https://tex.stackexchange.com/questions/394227/memoir-toc-indent-the-second-line-by-numberspace

% `\cftlastnumwidth` and these `\cftsetindents` are defined by the abntex2 class saying they obey
% the `ABNTEXsumario-abnt-6027-2012`.
% \newlength{\cftlastnumwidth}

\setlength{\cftlastnumwidth}{\cftsubsubsectionnumwidth}
\addtolength{\cftlastnumwidth}{-1em}

% http://www.tex.ac.uk/FAQ-tocloftwrong.html
% Use \setlength\cftsectionnumwidth{4em} to override all these values
\cftsetindents{part}         {0em}{\cftlastnumwidth}
\cftsetindents{chapter}      {0em}{\cftlastnumwidth}
\cftsetindents{section}      {0em}{\cftlastnumwidth}
\cftsetindents{subsection}   {0em}{\cftlastnumwidth}
\cftsetindents{subsubsection}{0em}{\cftlastnumwidth}
\cftsetindents{paragraph}    {0em}{\cftlastnumwidth}
\cftsetindents{subparagraph} {0em}{\cftlastnumwidth}



\addbibresource{modeloreferences.bib}


% When writing a large document, it is sometimes useful to work on selected sections of the document
% to speed up compilation time: https://en.wikibooks.org/wiki/TeX/includeonly
\newif\ifforcedinclude\forcedincludefalse
% \forcedincludetrue

% \addtoincludeonly{pretexto/agradecimentos}
% \addtoincludeonly{pretexto/epigrafe}
% \addtoincludeonly{pretexto/fichacatalografica}
% \addtoincludeonly{pretexto/folhadeaprovacao}
% \addtoincludeonly{pretexto/resumos}
% \addtoincludeonly{pretexto/siglas}
% \addtoincludeonly{pretexto/simbolos}

% \addtoincludeonly{chapters/chapter_1}
% \addtoincludeonly{chapters/chapter_2}
% \addtoincludeonly{chapters/chapter_3}
% \addtoincludeonly{chapters/chapter_4}
% \addtoincludeonly{chapters/chapter_5}
% \addtoincludeonly{chapters/chapter_6}
% \addtoincludeonly{chapters/conclusion}

% \addtoincludeonly{postexto/anexo_a}
% \addtoincludeonly{postexto/anexo_b}
% \addtoincludeonly{postexto/apendice_a}



% %----------------------------------------------------------------------------------------
% %   DOCUMENT CONTENTS
% %----------------------------------------------------------------------------------------
\doincludeonly
\begin{document}

    % Comment this after finishing the thesis, so you can start fixing the \flushbottom vs \raggedbottom
    % https://tex.stackexchange.com/questions/65355/flushbottom-vs-raggedbottom
    \raggedbottom

    % Retira espaço extra obsoleto entre as frases `Double space between sentences`
    % https://tex.stackexchange.com/questions/4705/double-space-between-sentences
    \frenchspacing

    % ELEMENTOS PRÉ-TEXTUAIS
    

% How to fix destination with the same identifier (name{page.A}) has been already used, duplicate ignored?
% https://tex.stackexchange.com/questions/386446/how-to-fix-destination-with-the-same-identifier-namepage-a-has-been-already
\hypersetup{pageanchor=false}




% ELEMENTOS PRÉ-TEXTUAIS
\includepdf{Imagens/FrenteCapaUFSC.pdf}%

\pretextual

% Capa
\imprimircapa

% Folha de rosto (o * indica que haverá a ficha bibliográfica)
\imprimirfolhaderosto*

% Inserir a ficha bibliografica
%
% Isto é um exemplo de Ficha Catalográfica, ou ``Dados internacionais de
% catalogação-na-publicação''. Você pode utilizar este modelo como referência.
% Porém, provavelmente a biblioteca da sua universidade lhe fornecerá um PDF
% com a ficha catalográfica definitiva após a defesa do trabalho. Quando estiver
% com o documento, salve-o como PDF no diretório do seu projeto e substitua todo
% o conteúdo de implementação deste arquivo pelo comando abaixo:
%
% \begin{fichacatalografica}
%    \includepdf{pretexto/ficha_catalografica.pdf}
% \end{fichacatalografica}


% http://portalbu.ufsc.br/ficha
% http://portal.bu.ufsc.br/servicos/ficha-de-identificacao-da-obra/
\begin{fichacatalografica}
    \vspace*{\fill}

    \begin{center}

        Catalogação na fonte pela Biblioteca Universitária da Universidade Federal de Santa Catarina.

        Arquivo compilado às \currenttime h do dia \today.

        \framebox[\textwidth]
        {
            \begin{minipage}{\textwidth}

                \ttfamily
                \imprimirautor

                \hspace{0.5cm} \imprimirtitulo : \imprimirsubtitulo, / \imprimirautor;
                orientador(a), \imprimirorientador;
                co-orientador(a) -- \imprimirlocal, \currenttime, \imprimirdata.

                % Prints how much pages there are on the document and links to the last page
                \hspace{0.5cm} \pageref{LastPage} p.
                \bigskip

                \hspace{0.5cm} \imprimirtipotrabalho -- \imprimirinstituicao,
                \imprimircentro, \imprimirprograma.
                \bigskip

                \hspace{0.5cm} Inclui referências
                \bigskip

                \hspace{0.5cm}
                    1. Uma Palavra-chave ~
                    2. Outra Palavra-chave ~
                    2. Mais Palavras-chave ~
                    I. \imprimirorientador ~
                    II. \imprimircoorientador ~
                    III. \imprimirprograma ~
                    IV. \imprimirtitulo ~
                \bigskip

                \hspace{7.75cm} CDU 02:141:005.7

            \end{minipage}
        }

    \end{center}

\end{fichacatalografica}



% Inserir errata

% Inserir folha de aprovação

% Isto é um exemplo de Folha de aprovação, elemento obrigatório da NBR
% 14724/2011 (seção 4.2.1.3). Você pode utilizar este modelo até a aprovação
% do trabalho. Após isso, substitua todo o conteúdo deste arquivo por uma
% imagem da página assinada pela banca com o comando abaixo:
% \includepdf{folhadeaprovacao_final.pdf}


\addtotextpreliminarycontent{\lang{Approval Sheet}{Folha de Aprovação}}

\begin{folhadeaprovacao}

    \begin{center}
        {\ABNTEXchapterfont\large\imprimirautor}

        \begin{center}
            \ABNTEXchapterfont\bfseries\Large\imprimirtitulo
        \end{center}

        \begin{minipage}{\textwidth}
            \lang
            {
                This thesis was considered appropriate to obtain the \imprimirformacao, in the area
                of \imprimirarea, and approved in its final form by the \imprimirprograma of the
                \imprimirinstituicao.
            }
            {
                Esta Tese foi julgada adequada para obtenção do Título de \imprimirformacao, na área
                de \imprimirarea, e aprovada em sua forma final pelo \imprimirprograma da
                \imprimirinstituicao.
            }
        \end{minipage}%

    \end{center}
    \begin{center}
        Florianópolis, \imprimirdata.
    \end{center}

    \assinatura{\textbf{Prof. Marcelo Lobo Heldwein, \lang{PhD.}{Dr.}} \\
    \lang{Coordinator of the}{Coordenador do} \imprimirprograma}

    \assinatura{\textbf{\imprimirorientador} \\ \imprimirorientadorRotulo \\
    \imprimirinstituicao~--~\imprimirinstituicaosigla}

    \assinatura{\textbf{\imprimircoorientador} \\ \imprimircoorientadorRotulo \\
    \imprimirinstituicao~--~\imprimirinstituicaosigla}

    \newpage
    \begin{flushleft}
        \textbf{\lang{Examination Board:}{Banca Examinadora:}}
    \end{flushleft}

    \assinatura{\textbf{Prof. Arnaldo José Perin, \lang{PhD.}{Dr.}} \\
    \imprimirinstituicao~--~\imprimirinstituicaosigla}

    \assinatura{\textbf{Prof. Denizar Cruz Martins, \lang{PhD.}{Dr.}} \\
    \imprimirinstituicao~--~\imprimirinstituicaosigla}

    \assinatura{\textbf{Prof. Roberto Francisco Coelho, \lang{PhD.}{Dr.}} \\
    \imprimirinstituicao~--~\imprimirinstituicaosigla}

    \assinatura{\textbf{Prof. Samir Ahmad Mussa, \lang{PhD.}{Dr.}} \\
    \imprimirinstituicao~--~\imprimirinstituicaosigla}

    \assinatura{\textbf{Prof. Telles Brunelli Lazzarin, \lang{PhD.}{Dr.}} \\
    \imprimirinstituicao~--~\imprimirinstituicaosigla}

\end{folhadeaprovacao}



% Dedicatória
\begin{dedicatoria}
    \vspace*{\fill}
    \centering
    \noindent
    \textit{ Este trabalho é dedicado às crianças adultas que,\\
        quando pequenas, sonharam em se tornar cientistas.} \vspace*{\fill}
\end{dedicatoria}

% Agradecimentos
\include{pretexto/agradecimentos}

% Epígrafe
\ifthenelse{ \equal{\WriteEpigraphs}{true} }{
\begin{epigrafe}

    \vspace*{\fill}
    \begin{flushright}
        \textit{``Assim como aquele pecado da juventude, este documento te perseguirá pelo resto da vida.''}\\Enio Valmor Kassick
    \end{flushright}
    \begin{flushright}
        \textit{``Estupidez trará mais autoconfiança do que o conhecimento e a bravura juntas.''}\\
        Adriano Ruseler
    \end{flushright}
    \begin{flushright}
        \textit{``Learn from yesterday, live for today, hope for tomorrow. The important thing is not to stop questioning.''}\\ Albert Einstein
    \end{flushright}
    \begin{flushright}
        \textit{``The true sign of intelligence is not knowledge but imagination.''}\\  Albert Einstein
    \end{flushright}
    \begin{flushright}
        \textit{``Peace cannot be kept by force; it can only be achieved by understanding.''}\\ Albert Einstein
    \end{flushright}
    \begin{flushright}
        \textit{``Whoever is careless with the truth in small matters cannot be trusted with important matters.''}\\    Albert Einstein
    \end{flushright}
    \begin{flushright}
        \textit{``Extraordinary claims require extraordinary evidence''}\\
        Carl Sagan
    \end{flushright}
    \begin{flushright}
        \textit{``Catholic, which I was until I reached the age of reason.''}\\
        George Carlin
    \end{flushright}
    \begin{flushright}
        \textit{``We made too many wrong mistakes.''}\\
        Yogi Berra
    \end{flushright}

\end{epigrafe}
}{
    % do something different..
}

% RESUMOS
%
% Ajusta o espaçamento dos parágrafos do resumo
\setlength{\absparsep}{18pt}
\swapcontents
{
    % Changing babel package inside a single chapter
    % https://tex.stackexchange.com/questions/20987/changing-babel-package-inside-a-single-chapter
    %
    % Multiple-language document - babel - selectlanguage vs begin/end{otherlanguage}
    % https://tex.stackexchange.com/questions/36526/multiple-language-document-babel-selectlanguage-vs-begin-endotherlanguage
    \addtotextpreliminarycontent{English's Abstract}
    \begin{otherlanguage*}{english}
    \begin{resumo}[Abstract]

        This is the english abstract.

        \imprimirpalavraschave{Keywords}{\begin{inparaitem}[]\palavraschaveingles\end{inparaitem}}

    \end{resumo}
    \end{otherlanguage*}
}
{
    \addtotextpreliminarycontent{Resumo em Português}
    \begin{otherlanguage*}{brazil}
    \begin{resumo}[Resumo]

        Segundo a \textcite[3.1-3.2]{NBR6028:2003}, o resumo deve ressaltar o
        objetivo, o método, os resultados e as conclusões do documento. A ordem e a extensão
        destes itens dependem do tipo de resumo (informativo ou indicativo) e do
        tratamento que cada item recebe no documento original. O resumo deve ser
        precedido da referência do documento, com exceção do resumo inserido no
        próprio documento. (\ldots) As palavras-chave devem figurar logo abaixo do
        resumo, antecedidas da expressão Palavras-chave:, separadas entre si por
        ponto e finalizadas também por ponto.

        Além disso, na UFSC o texto do resumo deve ser digitado, em um único bloco, sem espaço de parágrafo. O resumo deve
        ser significativo, composto de uma sequência de frases concisas, afirmativas e não de uma
        enumeração de tópicos. Não deve conter citações. Deve usar o verbo na voz passiva. Abaixo do
        resumo, deve-se informar as palavras-chave (palavras ou expressões significativas retiradas do
        texto) ou, termos retirados de thesaurus da área. 

        \imprimirpalavraschave{Palavras-chaves}{\begin{inparaitem}[]\palavraschaveportugues\end{inparaitem}}

    \end{resumo}
    \end{otherlanguage*}
}



% % resumo em francês
% \addtotextpreliminarycontent{Français Résumé}
% \begin{resumo}[Résumé]
%   \begin{otherlanguage*}{french}
%       Il s'agit d'un résumé en français.

%       \imprimirpalavraschave{Mots-clés}{latex. abntex. publication de textes.}
%   \end{otherlanguage*}
% \end{resumo}


% % resumo em espanhol
% \addtotextpreliminarycontent{Español Resumen}
% \begin{resumo}[Resumen]
%   \begin{otherlanguage*}{spanish}
%       Este es el resumen en español.

%       \imprimirpalavraschave{Palabras clave}{latex. abntex. publicación de textos.}
%   \end{otherlanguage*}
% \end{resumo}





% inserir lista de ilustrações
\pdfbookmark[0]{\listfigurename}{lof}
\listoffigures*
\cleardoublepage

% inserir lista de tabelas
\pdfbookmark[0]{\listtablename}{lot}
\listoftables*
\cleardoublepage

% inserir códigos fonte
% ---
\pdfbookmark[0]{\lstlistingname}{lol}
\lstlistoflistings*
\cleardoublepage

% inserir lista de abreviaturas e siglas


\addtothesispreliminarycontent{\chooselang{List of Acronyms}{Lista de Siglas}}

\begin{siglas}
    \item[ABNT] Associação Brasileira de Normas Técnicas
    \item[abnTeX] ABsurdas Normas para TeX
\end{siglas}



% Inserir lista de símbolos


\addtotextpreliminarycontent{\lang{List of Symbols}{Lista de Símbolos}}

% Devam aparecer na mesma ordem de ocorrência no texto.
\begin{simbolos}
    \item[$\arcs$]{Conjunto dos arcos $\in \graph$}
  \item[$\arrivalTime_\request$] Instante de chegada do pedido $\request$
    \item[$\requestLatestArrivalTime$]
      Último instante possível para a chegada do pedido $\request$
    \item[$\pickupNodes$] Subconjunto dos nós, contendo apenas nós de coleta
    \item[$\arcCost{i}{j}$] Custo de viagem entre os nós $i$ e $j$
    \item[$\deliveryNodes$] Subconjunto dos nós, contendo apenas nós de entrega
    \item[$\dynamism$] Dinamismo
    \item[$\earliestTimeWindow_{\originIndex}$]
      Limite inferior da janela de tempo da coleta do pedido $\request$
    \item[$\earliestTimeWindow_{\destinationIndex}$]
      Limite inferior da janela de tempo da entrega do pedido $\request$
    \item[$\graph$] Grafo
    \item[$\planingHorizon$] Limite superior do horizonte de planejamento
    \item[$\request$] Um elemento do conjunto $\requests$
    \item[$\vehiclesSet$] Conjunto de veículos
    \item[$\vehicle$] Um elemento do conjunto $\vehiclesSet$
    \item[$\vehiclesSetSize$] Tamanho do conjunto de veículos
    \item[$\maxRideTime$] Tempo máximo de viagem global
    \item[$\maxRideTime_\request$] Tempo máximo de viagem do pedido 
      $\request$
    \item[$\latestTimeWindow_{\originIndex}$]
      Limite superior da janela de tempo da coleta do pedido $\request$
    \item[$\latestTimeWindow_{\destinationIndex}$]
      Limite superior da janela de tempo da entrega do pedido $\request$
    \item[$\numberOfRequests$] Número de pedidos de uma instância
    \item[$\requests$] Conjunto dos pedidos de transporte 
    \item[$\capacity$] Capacidade global
    \item[$\vehicleCapacity$] Capacidade do veículo $\vehicle$
    \item[$\load$] Carregamento global
    \item[$\requestLoad$] Carregamento do pedido $\request$
    \item[$\originNodeLoad$] Carregamento do pedido no nó $\startNode$
    \item[$\destinationNodeLoad$]
     Carregamento do pedido no nó $\destinationNode$
    \item[$\nodeServiceTime{\originIndex}$]
      Tempo de serviço no nó $\request$
    \item[$\maxRouteTime$] Tempo de rota máximo global
    \item[$\vehicleMaxRouteTime$] Tempo de rota máximo do veículo $\vehicle$
    \item[$\arcTravelTime{i}{j}$] Tempo de viagem entre os nós $i$ e $j$
    \item[$\urgency$] Urgência
    \item[$\nodes$] Conjunto dos nós $\in \graph$
    \item[$\startNode$] Nó inicial
    \item[$\lastNode$] Nó final 
    \item[$\originNode$] O nó de coleta do pedido $\request$
    \item[$\destinationNode$] O nó de entrega do pedido $\request$
    \item[$\timeWindowWidth$] Largura global das janelas de tempo 
    \item[$\staticPercentage$] Percentagem de pedidos estáticos
    \item[$\maneuverTime$] Tempo de manobra
    \item[$\intervalsBetweenArrivals$]
      Lista de intervalos entre instantes de chegada de pedidos 
      consecutivos
    \item[$\intervalBetweenArrivals_\request$]
      Intervalo de tempo entre os pedidos $i + 1$ e $i$
    \item[$\perfectInterval$] Intervalo perfeito entre instantes de chegada
    \item[$\deviationFromPerfectInterval_\request$]
      Diferença entre o $\perfectInterval$ e 
      $\intervalBetweenArrivals_\request$
    \item[$\normalDistribution{a}{b}$]
       Distribuição normal entre os valores $a$ e $b$
    \item[$\uniformDistribution{a}{b}$]
      Distribuição uniforme entre os valores $a$ e $b$
\end{simbolos}


% Inserir o sumario
\pdfbookmark[0]{\contentsname}{toc}
\tableofcontents*
\cleardoublepage




% How to fix destination with the same identifier (name{page.A}) has been already used, duplicate ignored?
% https://tex.stackexchange.com/questions/386446/how-to-fix-destination-with-the-same-identifier-namepage-a-has-been-already
\hypersetup{pageanchor=true}








    % ELEMENTOS TEXTUAIS
    \textual

    % To automatically put a [Go To Top/Back] ←← | ← on each section
    \addGoToSummary

    \setlength\beforechapskip{50pt}
    \setlength\midchapskip{20pt}
    \setlength\afterchapskip{20pt}

    % Introdução (exemplo de capítulo sem numeração, mas presente no Sumário)
    \include{chapters/chapter_1}

    % PARTE
    \ifforcedinclude\else\part{Preparação da pesquisa}\fi

    % Capitulo com exemplos de comandos inseridos de arquivo externo
    \include{chapters/chapter_2}

    % PARTE
    \ifforcedinclude\else\part{Referenciais teóricos}\fi

    % Capitulo de revisão de literatura
    
% The \phantomsection command is needed to create a link to a place in the document that is not a
% figure, equation, table, section, subsection, chapter, etc.
\cleardoublepage
\phantomsection


% \MakeUppercase in \section and \chapter with hyperref cause trouble
% https://tex.stackexchange.com/questions/199374/makeuppercase-in-section-and-chapter-with-hyperref-cause-trouble
\addcontentsline{toc}{chapter}{\HyperrefUppercase\chooselang{Chapter 3}{Capítulo 3}}


\chapter{Lorem ipsum dolor sit amet}

\begin{flushright}
    \showfont
\end{flushright}

\newpage


% ---
\section{Aliquam vestibulum fringilla lorem}
% ---

Lipsum me [1]

Lipsum me [2-3]


    % PARTE
    \ifforcedinclude\else\part{Resultados}\fi

    % Primeiro capitulo de Resultados
    
% The \phantomsection command is needed to create a link to a place in the document that is not a
% figure, equation, table, section, subsection, chapter, etc.
\cleardoublepage
\phantomsection


% \MakeUppercase in \section and \chapter with hyperref cause trouble
% https://tex.stackexchange.com/questions/199374/makeuppercase-in-section-and-chapter-with-hyperref-cause-trouble
\addcontentsline{toc}{chapter}{\HyperrefUppercase{Capítulo 4}}


\chapter{Lectus lobortis condimentum}
% ---

\begin{flushright}
    \showfont
\end{flushright}

\newpage

% ---
\section{Vestibulum ante ipsum primis in faucibus orci luctus et ultrices
    posuere cubilia Curae}
% ---

\lipsum[21-22]

\showfont


    % Segundo capitulo de Resultados
    
% The \phantomsection command is needed to create a link to a place in the document that is not a
% figure, equation, table, section, subsection, chapter, etc.
\cleardoublepage
\phantomsection


% \MakeUppercase in \section and \chapter with hyperref cause trouble
% https://tex.stackexchange.com/questions/199374/makeuppercase-in-section-and-chapter-with-hyperref-cause-trouble
\addcontentsline{toc}{chapter}{\HyperrefUppercase{Capítulo 5}}


\chapter[Nam sed tellus sit amet lectus]{Nam sed tellus sit amet lectus urna ullamcorper tristique interdum
elementum}
% ---

\begin{flushright}
\showfont
\end{flushright}

\newpage



\section[Some encoding tests]{\showfont}
\subsection{\showfont}
\subsubsection{\showfont}
\subsubsubsection{\showfont}


% ---
\section{Pellentesque sit amet pede ac sem eleifend consectetuer}
% ---

Lipsum me [24-26]


    % Terceiro capitulo de Resultados
    
% The \phantomsection command is needed to create a link to a place in the document that is not a
% figure, equation, table, section, subsection, chapter, etc.
%
% When do I need to invoke \phantomsection?
% https://tex.stackexchange.com/questions/44088/when-do-i-need-to-invoke-phantomsection
\phantomsection


% Is it possible to keep my translation together with original text?
% https://tex.stackexchange.com/questions/5076/is-it-possible-to-keep-my-translation-together-with-original-text
\begin{braziltext}

    \chapter[Simulações]{Simulações GEP + REP}


    \section{Retificador Elétrico de Propulsão}


    \section{Simulação do sistema completo}


    \subsection{Operação em condições nominais}


    \subsection{Operação em condições de falha}


    \subsubsection{Diodo em curto-circuito}


    \subsubsection{``Abertura'' de fusível}


    \section{Considerações Finais}

\end{braziltext}


\begin{englishtext}

    \chapter[Simulations]{Simulations GEP + REP}


    \section{Rectifier Electric Propulsion}


    \section{Full system simulation}


    \subsection{Operation under nominal conditions}


    \subsection{Operation under fault conditions}


    \subsubsection{Short-circuit diode}


    \subsubsection{``Opening'' of fuse}


    \section{Final considerations}

\end{englishtext}



    % Finaliza a parte no bookmark do PDF
    % para que se inicie o bookmark na raiz
    % e adiciona espaço de parte no Sumário
    \phantompart

    % Conclusão (outro exemplo de capítulo sem numeração e presente no sumário)
    

% The \phantomsection command is needed to create a link to a place in the document that is not a
% figure, equation, table, section, subsection, chapter, etc.
%
% When do I need to invoke \phantomsection?
% https://tex.stackexchange.com/questions/44088/when-do-i-need-to-invoke-phantomsection
\phantomsection

% ---
% Considerações Finais (outro exemplo de capítulo sem numeração e presente no sumário)
% ---
\chapter*[]{\lang{Final Remarks}{Considerações Finais}}
\phantomsection
\addcontentsline{toc}{chapter}{Considerações Finais}%

    Lipsum me [31-33]






    % ELEMENTOS PÓS-TEXTUAIS
    \postextual
    \setlength\beforechapskip{0pt}
    \setlength\midchapskip{15pt}
    \setlength\afterchapskip{15pt}

    % Referências bibliográficas
    \begingroup
        % Using BibTeX to make a list of references without having citations in the body of the document?
        % https://tex.stackexchange.com/questions/17128/using-bibtex-to-make-a-list-of-references-without
        % \nocite{*}

        % How to modify line spacing per entry of bibliography?
        % https://tex.stackexchange.com/questions/163559/how-to-modify-line-spacing-per-entry-of-bibliography
        \linespread{1.18}\selectfont
        \printbibliography
    \endgroup

    % Glossário, consulte o manual da classe abntex2 para orientações sobre o glossário.
    % \ifforcedinclude\else\glossary\fi

    % Apêndices, inicia os apêndices
    \ifforcedinclude\else\begin{apendicesenv}
        % Imprime uma página indicando o início dos apêndices
        \partapendices

        \setlength\beforechapskip{50pt}
        \setlength\midchapskip{20pt}
        \setlength\afterchapskip{20pt}

        \include{postexto/apendice_a}
    \end{apendicesenv}\fi

    % Anexos, inicia os anexos
    \ifforcedinclude\else\begin{anexosenv}
        % Imprime uma página indicando o início dos anexos
        \partanexos

        \setlength\beforechapskip{50pt}
        \setlength\midchapskip{20pt}
        \setlength\afterchapskip{20pt}

        \include{postexto/anexo_a}
        \include{postexto/anexo_b}
    \end{anexosenv}\fi

    % INDICE REMISSIVO
    \ifforcedinclude\else
        \phantompart
        \printindex
    \fi

\end{document}

