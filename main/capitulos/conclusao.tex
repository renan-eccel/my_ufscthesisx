\chapter{Conclusão}\label{ch:conclusao}

% TODO Rodrigo: qual a importância de dinamismo e urgência?

Este documento apresentou, de forma detalhada, conjuntos de instâncias de 
\textit{benchmark} para o DDARP e o DPDPTW e analisou os métodos usados para a 
distribuição temporal dos instantes de chegadas dos pedidos usando, para isso,
as medidas de dinamismo e urgência propostas por 
\citeonline{van_lon_measures_2016}.

Através dessa análise, observou-se que os conjuntos possuem uma distribuição 
acumulada em intervalos de valores de dinamismo e urgência, principalmente 
devido ao baixo número de instâncias, ao uso de métodos de dinamização simples,
devido ao uso de instâncias estáticas com janelas de tempo de coleta
acumuladas no início do horizonte de planejamento e da correlação entre os
limites inferiores das janelas de tempo de coleta e os instantes de chegada dos
pedidos.  

Portanto, as instâncias de \textit{benchmark} apresentadas neste documento
devem ser usadas com cautela em experimentos, tendo em vista que 
essas características podem enviesar a comparação de dois algoritmos distintos.

Espera-se que este trabalho sirva como base referencial dos conjuntos de 
\textit{benchmark} disponíveis assim como algumas de suas características
temporais. 
Com isso, os demais pesquisadores da área de roteamento dinâmico de 
veículos que tenham interesse em estudar o comportamento de algoritmos de 
solução para o DDARP e DPDPTW através de simulações computacionais de cenários 
diversificados podem consultar este documento para fazer uma escolha mais
consciente das instâncias que irão usar.
Todos os dados das instâncias estudadas neste artigo estão disponíveis para 
consulta e utilização, assim como todos os códigos usados para a análise das 
instâncias \cite{eccel_problemas_2019}.

Para trabalhos futuros, recomenda-se a aplicação dos métodos de dinamização
estudados em diferentes instâncias estáticas, desse modo possibilitando uma
melhor comparação do que é influência gerada pelo próprio método e o que é 
gerado pelas características das instâncias estáticas.
Outra proposta interessante é uma análise dos fatores espaciais das 
instâncias, com relação à distribuição dos locais de coleta e entrega dos 
pedidos.
