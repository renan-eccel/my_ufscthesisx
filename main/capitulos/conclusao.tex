\chapter{Conclusão}\label{ch:conclusao}

% TODO Rodrigo: qual a importância de dinamismo e urgência?

Este documento apresentou, de forma detalhada, conjuntos de instâncias de 
\textit{benchmark} para o DDARP e o DPDPTW e analisou os métodos usados para a 
distribuição temporal dos instantes de chegadas dos pedidos usando, para isso,
as medidas de dinamismo e urgência propostas por 
\citeonline{van_lon_measures_2016}.

Através dessa análise, observou-se que os conjuntos possuem pouca 
variabilidade em relação aos valores de dinamismo e urgência, principalmente 
devido ao baixo número de instâncias, ao uso de métodos de dinamização simples 
e devido ao uso de instâncias estáticas com janelas de tempo de coleta
acumuladas no início do horizonte de planejamento.

Espera-se que este trabalho sirva de base para demais pesquisadores da área de 
roteamento dinâmico de veículos que tenham interesse de estudar o comportamento
de algoritmos de solução para o DDARP e DPDPTW através de simulações 
computacionais de cenários diversificados.
Todos os dados das instâncias estudadas neste artigo estão disponíveis para 
consulta e utilização, assim como todos os códigos usados para a análise das 
instâncias \cite{eccel_problemas_2019}.

Para trabalhos futuros, recomenda-se a aplicação dos métodos de dinamização
estudados em diferentes instâncias estáticas, desse modo possibilitando uma
melhor comparação do que é influência gerada pelo próprio método e o que é 
gerado pelas características das instâncias estáticas.
Outra proposta interessante é uma análise dos fatores espaciais das 
instâncias, com relação à distribuição dos locais de coleta e entrega dos 
pedidos.
