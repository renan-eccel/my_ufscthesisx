\chapter{Introdução}\label{ch:introducao}

Problemas de roteamento dinâmico de veículos são objetos de pesquisa há cerca 
de três décadas \cite{psaraftis_dynamic_2015}.
Derivados de problemas de roteamento de veículos (VRP - \textit{Vehicle Routing
Problem}) clássicos, como o problema \textit{dial-a-ride} (DARP -
\textit{Dial-A-Ride Problem}) e o problema de coleta e entrega com
janelas de tempo (PDPTW - \textit{Pickup and Delivery Problem with 
Time Windows}), os problemas dinâmicos buscam modelar casos em que um 
ou mais parâmetros do problema não são totalmente conhecidos 
\textit{a priori} e podem variar durante o período de operação. 

Dentre o grupo de problemas dinâmicos de roteamento veicular, o problema 
\textit{dial-a-ride} dinâmico (DDARP - \textit{Dynamic Dial-A-Ride Problem}) 
\cite{psaraftis_dynamic_1988} e o problema dinâmico de coleta e entrega com
janelas de tempo (DPDPTW - \textit{Dynamic Pickup and Delivery Problem with 
Time Windows}) \cite{dumas_1991} são de grande interesse para o 
desenvolvimento de novas tecnologias de transporte urbano. 
São esses os problemas que precisam ser solucionados quando precisa-se 
de um serviço de transporte compartilhado dinâmico (\textit{dynamic 
ride-sharing}) \cite{agatz_optimization_2012, 
alonso-gonzalez_potential_2018}, ou quando necessita-se 
fazer uma entrega imediata de encomendas \cite{pankratz_dynamic_2005}.
Atualmente, algumas empresas fornecem serviços desse tipo (Uber Pool, Uber
Eats, Rappi).
Entretanto, com o esperado avanço tecnológico na área de veículos
autônomos e a diversificação dos transportes públicos introduzida
principalmente por sistemas de mobilidade como um serviço (MaaS -
\textit{Mobility as a Service}), serão cada vez mais necessário algoritmos 
para a solução dos problemas DDARP e DPDPTW em menor tempo e proporcionando 
um melhor resultado \cite{fulton_three_2017}. 

Para que avanços científicos ocorram, se faz necessária a montagem
de experimentos para comparação direta de dois ou mais algoritmos. 
Essa é uma tarefa extremamente técnica, e que muitas vezes é
desconsiderada e ocultada em muitos artigos científicos.

Um experimento científico na área de algoritmos para soluções de problemas 
de roteamento dinâmico normalmente requer três componentes: um simulador, já 
que testes empíricos são economicamente inviáveis para esse tipo de pesquisa 
\cite{maciejewski_towards_2017}; um ou mais algoritmos que resolvam o problema 
e um conjunto de dados que determina os cenários a serem simulados, intitulados
instâncias.
Com isso é possível gerar resultados de eficiência e de tempo computacional dos
métodos de solução dos problemas para, então, compara-los com demais métodos
expostos na literatura.

Entretanto, como apontado por \textcite{merali_computational_2010}, atualmente
a computação e as ferramentas de programação estão muito mais complexas que 
antigamente, fazendo com que os cientistas, normalmente não preparados para 
esse tipo de tarefa, tenham dificuldades em implementar códigos em um tempo 
hábil, livres de erros, corretamente testados e bem documentados.

Portanto, é crucial que todo código e dado utilizado ou produzido durante uma
pesquisa científica deve ser publicado e tornado disponível para todos com
licença de código aberto \cite{ince_case_2012}.
Esta prática permite a validação do código usado e a reprodução do experimento,
garantindo os resultados científicos encontrados e permitindo com que o
código seja reutilizado para a experimentação de demais métodos.
Além disso, facilita a manutenção e melhoramento do código para futuros usos,
evitando o que \textcite{merali_computational_2010} intitulou de ``código
monstro'' (``monster code''), códigos legado que foram alterados sem nenhum
cuidado no decorrer dos anos e se tornaram conjuntos de arquivos mal 
documentados e mal escritos.

Sabe-se que a difusão do compartilhamento de informações relacionadas aos
códigos e dados de experimentação requerer uma drástica mudança na cultura de 
publicações acadêmicas \cite{leveque_reproducible_2012}.
Atualmente, muitas das áreas de pesquisas não possuem boas práticas de 
compartilhamento de dados. 
A área de roteamento dinâmico de veículos não é diferente, porém existem
exceções.

Apesar de pouco usados já existem alguns simuladores de código aberto que 
comportam a modelagem de problemas de roteamento dinâmico 
\cite{maciejewski_towards_2017, van_lon_rinsim:_2012, mayer_open-source_2016}.
Também existem casos de experimentos completos disponíveis para reprodução e
reuso \cite{van_lon_when_2017, van_lon_towards_2015, van_lon_measures_2016}

Entretanto, para que os resultados possam ser comparados entre artigos, 
necessita-se que as simulações conduzidas usem os mesmos cenários de 
simulação (instâncias).
Na área de VRP estáticos é comum a existência de conjuntos de cenários
canônicos extensivamente usados que facilitam a comparação entre 
algoritmos \cite{mendoza_vrp-rep:_2014}, chamados de
conjunto de instâncias de \textit{benchmark}.

Lamentavelmente, na área de roteamento dinâmico de veículos não existem 
instâncias de \textit{benchmark} de referência que sejam amplamente usadas 
\cite{pillac_review_2013, maciejewski_towards_2017}. 
Isso faz com que a comparação entre algoritmos
propostos para resolver problemas como DPDPTW e DDARP seja difícil e custosa.

O objetivo deste artigo é compilar as informações disponíveis sobre as
instâncias de \textit{benchmark} de DDARPs e DPDPTWs que estejam acessíveis e
disponíveis para uso, com enfoque na forma com que são feitas a distribuição
dos pedidos ao longo do período de operação dos sistemas.
Também faz parte do escopo uma análise de duas medidas, urgência e 
grau de dinamismo, propostas por \citeonline{van_lon_measures_2016} que ajudam 
a identificar as características temporais das instâncias. 
Com isso, procura-se auxiliar o processo de busca e seleção de conjuntos de 
instâncias interessantes para uso em simulações e teste de novos algoritmos ou,
caso necessário, auxiliar a seleção de um método de geração de instâncias
dinâmicas adequado ao interesse de pesquisa.

% as linhas contidas entre \iffalse e \fi estão todas comentadas
% elas descrevem em mais detalhes a dificuldades de fazer uma pesquisa na área
% de algoritmos para problemas de roteamento, assim como algumas "dicas" para
% quem quer fazer o mesmo.
% Essa parte se encontra mal escrita, no entanto, pode ser interessante 
% adicionar algo dela na introdução 
\iffalse
Primeiramente, se faz necessário a utilização de um simulador para rodar
testes, já que testes empíricos são inviáveis para esse tipo de pesquisa.
Por muitos anos os pesquisadores optaram por implementar seus próprios 
simuladores e não disponibilizar o código de maneira aberta, isso fez com que a
comunidade perdesse uma grande oportunidade de cooperação.
Com um código de livre acesso outros pesquisadores possuem a possibilidade de
reutilizar o código, facilitando as implementações de pesquisas na área, além
de poder revisar, melhorar ou estender o código. 
Atualmente, apesar de não serem amplamente usados, já existem simuladores de 
problemas de roteamento dinâmico com código aberto disponíveis para livre uso
acadêmico (\cite{van_lon_rinsim:_2012}; \cite{mayer_open-source_2016};
\cite{maciejewski_towards_2017})

Após a escolha de um simulador já disponível ou a implementação de um novo, se
faz necessária a implementação do algoritmo a ser testado. 
É nessa etapa que o pesquisador aplica seus conhecimentos técnicos para a 
implementação do seu objeto de pesquisa.

Após essa implementação precisa-se então da comparação de performance entre
o novo algoritmo e os algoritmos propostos anteriormente.
Para isso existem dois métodos.

Um deles é o pesquisador implementar os algoritmos com base na descrição dada 
nos artigos publicados, tarefa que, normalmente, toma bastante tempo do 
pesquisador e não agrega nenhum valor.
Atualmente, já existem pesquisadores que montam seus experimentos e deixam
disponíveis para acesso, replicação e utilização (\cite{van_lon_towards_2015};
\cite{van_lon_measures_2016}; \cite{van_lon_when_2017}).

Uma outra forma de comparação de algoritmos é possível quando os experimentos
são baseados em cenários cujos dados estão disponíveis para todos e
que já foram utilizados para testes em algoritmos de outros artigos.
A esse conjunto de dados se dá o nome de conjunto de instâncias de
\textit{benchmark}.
Portanto, quando usado um conjunto de instâncias de \textit{benchmark}, os
resultados expostos nos artigos podem ser comparados diretamente, sem a
necessidade de replicação dos experimentos.
Entretanto, na área de roteamento dinâmico de veículos não existem instâncias
de \textit{benchmark} de referência que sejam amplamente usadas
\cite{pillac_review_2013}

O objetivo deste artigo é compilar as informações disponíveis sobre as
instâncias de \textit{benchmark} de DDARPs e DPDPTWs que estejam acessíveis e
disponíveis para uso. Também faz parte do escopo a confecção de uma tabela
comparando as principais características dos conjuntos de \textit{benchmark}
encontrados, assim como uma análise usando duas medidas propostas por
\textcite{van_lon_measures_2016} que ajudam a identificar as características
temporais das instâncias.
\fi

O Capítulo~\ref{ch:formulacao_problemas} apresenta as formulações dos problemas
de interesse.
Logo em seguida, no Capítulo~\ref{ch:instancias} os conjuntos de instâncias de 
\textit{benchmark} são descritos.
O Capítulo~\ref{ch:medidas} contém a definição das métricas usadas para a 
avaliação dos conjuntos de instâncias de \textit{benchmark} e o 
Capítulo~\ref{ch:analise} apresenta estas análises.
Por último, o Capítulo~\ref{ch:conclusao} encerra descrevendo as conclusões  
e recomendações para futuras pesquisas.
