\chapter{Formulação dos Problemas}\label{ch:formulacao_problemas}
Neste documento o termo problema refere-se a um problema de otimização.
Neste tipo de problemas o objetivo é encontrar a melhor solução dentre o
conjunto de todas as soluções factíveis.
Esses problemas podem ser representados matematicamente através de um conjunto
de equações que descrevem uma função objetivo e um conjunto de restrições que
determinam o espaço de soluções.
A esta representação matemática se dá o nome de modelo, ou formulação.

O modelo de um problema é feito de forma abstrata, ou seja, ele não
contém nenhum dado numérico relativo à realidade. 
Sua formulação é baseadas em parâmetros não definidos.
Podemos ver como exemplo um problema de divisão.
Podemos modelar o problema de divisão como sendo:
%
\begin{equation}
  f(x, y) = x \div y.
\end{equation}

\noindent Os parâmetros abstratos são $x$ e $y$, e nenhum valor 
numérico é dado a eles.
Quando define-se os parâmetros de um modelo tem-se então uma instância do
problema.
Seguindo o exemplo, uma instância do problema de divisão poderia ser dada por:
%
\begin{equation}
  f(15, 2) = 15 \div 2.
\end{equation}

\noindent Um problema pode ter mais de um modelo ou formulação, assim como cada
modelo pode possuir inúmeras instâncias.

O restante desse capítulo define uma formulação para o DDARP e o DPDPTW. 
Para isso, inicialmente será definido o DARP usando como base a formulção 
apresentada por \textcite{cordeau_tabu_2003}. 
Após isso, expande-se a definição do DARP para o PDPTW. 
Por fim, define-se o que seria dinamismo para estes dois problemas, para então 
apresentar as definições do DDARP e do DPDPTW.

\section{Formulação do DARP}\label{sec:formulacao_DARP}
O DARP consiste em um conjunto de pedidos de transporte de passageiros entre 
diferentes locais de coleta e entrega que devem ser atendidos por uma frota
de veículos com capacidade para levar mais de um passageiro concomitantemente. 
O objetivo é então encontrar um conjunto de rotas para os veículos da frota
que minimize o tempo e/ou o custo para completar todos os pedidos de
transporte.

Cada um desses pedidos possui um local de coleta e uma janela de
tempo associada que identifica os limites superiores e inferiores de tempo no 
qual o usuário deseja ser coletado para viagem.
De maneira análoga, o pedido de transporte também possui um ponto de destino do
passageiro e janela de tempo para entrega.
Além disso, os passageiros desejam chegar em seus destinos sem que pra isso
precisem viajar por muito tempo.
Ou seja, apesar de ter definido uma janela de tempo para o início e o fim
desejados de sua viagem, o passageiro também espera que seu trajeto não demore
mais que o que ele considera necessário.

Para servir o conjunto de pedidos de viagens anteriormente citado, uma frota de
veículos com capacidade de transportar diversos passageiros, de maneira
concomitante, é posta à disposição.
Quanto aos veículos, estes possuem também uma restrição de tempo de rota máxima
devido a suas limitações com relação a combustível ou tempo de jornada do
motorista.

O problema então é gerar uma rota para cada veículo da frota, de maneira a
conseguir completar todos os pedidos de viagem, respeitando as janelas de tempo
de entrega e coleta, o tempo de viagem considerado satisfatório
pelos usuários, assim como a restrição de tempo máximo de rota pra cada
veículo.

Portanto, podemos definir o DARP através das seguintes considerações:
Sendo $\numberOfRequests$ o número de pedidos a serem servidos.
O DARP pode ser definido por um grafo direcionado completo 
$\graph(\nodes,\arcs)$, em que $\nodes$ são os nós e $\arcs$ são os arcos do
grafo, 
$\nodes = \pickupNodes \cup \deliveryNodes \cup \{\startNode, \lastNode \}$ com
$\pickupNodes = \{\node{1}, \ldots, \node{\numberOfRequests}\}$, e 
$\deliveryNodes = \{\node{\numberOfRequests + 1}, \ldots,
\node{2\numberOfRequests}\}$.
Os subconjuntos $\pickupNodes$ e $\deliveryNodes$ contêm, respectivamente, 
os nós de coleta e entrega dos pedidos, enquanto os nós $\startNode$ e 
$\lastNode$ representam os nós de origem e destino dos veículos.
Todos os veículos da frota devem iniciar suas rotas no nó $\startNode$ e
finalizá-las no nó $\lastNode$.
Para cada pedido $\request \in \requests = \{\request, \ldots,
\numberOfRequests\}$ temos associado um nó de origem $\originNode$ e um 
nó de destino $\destinationNode$.
A cada arco $\arc{\node{i}}{\node{j}} \in \arcs$ é associado um custo $\arcCost{i}{j}$ 
e um tempo de viagem $\arcTravelTime{i}{j}$.

Cada veículo $\vehicle \in \vehiclesSet$, sendo $\vehiclesSet$ o conjunto de 
veículos disponíveis, possui uma capacidade $\vehicleCapacity$ e um tempo 
máximo total de rota $\vehicleMaxRouteTime$.
Para cada nó $\node{i} \in \nodes$ existe um carregamento $\nodeLoad{i}$ 
associado e um tempo de serviço $\nodeServiceTime{i}$, não negativo, sendo que 
$\startNodeServiceTime = \lastNodeServiceTime = 0$, 
$\originNodeLoad = -\destinationNodeLoad$. 

As janelas de tempo dos pontos de coleta e entrega de cada pedido podem ser
definidas por $[\earliestTimeWindow_{i},\latestTimeWindow_{i}]$ e são 
associadas aos nós $\node{i} \in \nodes$, em que $\earliestTimeWindow_{i}$ e 
$\latestTimeWindow_{i}$ representam, respectivamente, o limite inferior e 
superior para o instante de tempo que o serviço deve começar no nó $\node{i}$.
Define-se também $\timeWindowWidth_{i} = \latestTimeWindow_{i} 
- \earliestTimeWindow_{i}$ como o tamanho das janelas de tempo 
de cada um dos nós. Denota-se por $\maxRideTime_\request$ o tempo máximo de 
viagem de um pedido, limitado pelo valor de tempo que o passageiro considera 
aceitável para o seu trajeto. 

Finalmente, define-se um intervalo de tempo, denominado horizonte de
planejamento, $[0, \planingHorizon]$, no qual o instante $0$ (zero) representa
o início da operação, em que todos os veículos estão localizados no nó inicial 
($\startNode$) e nenhum outro ponto do grafo foi visitado.
Por conseguinte, o instante $\planingHorizon$ representa o fim da operação, 
em que os veículos terminaram de cumprir suas rotas, levando todos os usuários 
dos seus respectivos pontos iniciais para os pontos finais, e se 
encontram no nó final ($\lastNode$). Todas as janelas de tempo dos nós de
coleta e entrega ($[\earliestTimeWindow_i, \latestTimeWindow_i], 
\forall \node{i} \in \pickupNodes \cup \deliveryNodes$) devem estar contidas 
no intervalo de tempo $[0, \planingHorizon]$.

\section{Formulação do PDPTW}\label{sec:formulacao_PDPTW}

O PDPTW é um problema de roteamento de pedidos muito semelhante ao DARP.
Como no DARP, o PDPTW também possui um conjunto de pedidos de transporte com
origens e destinos diferentes e janelas associadas.
Também conta com uma frota de veículos com capacidade de transportar mais de um
pedido de maneira concomitante.
Entretanto, os pedidos contidos em um PDPTW são referentes ao transporte de
mercadorias e não de passageiros.
É por esse motivo que surge a única diferença entre a formulação do DARP e do
PDPTW \cite{parragh_survey_2008}.

Na formulação do DARP apresentada na Seção~\ref{sec:formulacao_DARP} definiu-se
um parâmetro $\maxRideTime_\request$ para representar o tempo máximo de viagem
de um pedido, o qual limita o tempo total que um passageiro deseja permanecer
dentro do veículo.
Entretanto, para o PDPTW essa restrição não é necessária. Como a carga não
sofre nenhum desconforto com a demora no tempo de viagem, pode-se considerar 
que a formulação apresentada anteriormente para o DARP pode ser usada também 
para o PDPTW, porém, para esse último, o parâmetro 
$\maxRideTime_\request = \infty, \forall \request \in \requests$.

% bloco comentado
% explica as diferenças entre os problemas DARP e PDPTW
% TODO: Verificar a necessidade de adição de alguma informação presente neste
%       bloco
\iffalse
Entretanto, apesar da formulação do DARP e do PDPTW ser similar, suas
instâncias não necessariamente são.
Como citado anteriormente, cada um desses problemas trata de um tipo de carga
diferente e por esse motivo, os pedidos das instâncias do PDPTW podem tomar
formas diferentes quando comparados com os pedidos de transporte de passageiros
apresentados nas instâncias DARP.

Pega-se como exemplo as janelas de tempo de coleta e de entrega.
Em um cenário de transporte de passageiros, normalmente essas janelas são
restritas a intervalos de tempo pequenos.
Os usuários do sistema de transporte possuem uma flexibilidade no tempo de
início e fim de viagem, porém limitada por outras atividades que o
usuário tenha agendada em seu dia, como reuniões, início e fim da jornada de
trabalho, início e fim dos horários do colégio, entre outros.
Já no caso do transporte de cargas, como é modelado no PDPTW, esses tipo de
restrição existem em menor escala.
Por isso as janelas de tempo de coleta e entrega costumam ser maiores em
instâncias PDPTW que em instâncias DARP.

Outra característica que difere as instâncias PDPTW de instâncias DARP é a
relação entra a capacidade dos veículos e o tamanho da carga de cada pedido.
Para as instâncias DARP, normalmente os pedidos e a capacidade dos veículos 
são, ambos, valores na ordem de grandeza $10^1$.
Já para as instâncias PDPTW, as ordens de grandeza desses dois parâmetros podem
ser bastante distintas.
Isso ocorre pois as características das cargas transportadas podem variar
bastante de instância a instância.
\textcite{gendreau_neighborhood_2006}, por exemplo, usa instâncias em que a
capacidade do veículo é infinita pois representa o transporte de cartas e
envelopes cujo espaço ocupado dentro do veículo é ínfimo e portanto 
desconsiderável.
\fi

\section{Formulação do DDARP e do DPDPTW}

Nas formulações do DARP e do PDPTW, apresentadas nas
Seções~\ref{sec:formulacao_DARP}~e~\ref{sec:formulacao_PDPTW}, respectivamente,
os pedidos são totalmente conhecidos antes de resolver o problema e não variam 
durante a operação das rotas determinadas pela solução do problema, ou seja, 
são problemas estáticos \cite{psaraftis_dynamic_1988}.

Em suma, \textcite{psaraftis_dynamic_2015} apontam que, 
todo problema de roteamento que requer a determinação de um conjunto
de rotas pré-planejadas que não serão alteradas durante sua execução e que 
são calculadas com dados que não evoluem em tempo real é considerado estático.
Do contrário, todo problema de roteamento que receba dados atualizados em tempo
real concomitantemente com o processo de determinação da rota é dito um
problema dinâmico. 

Os problemas dinâmicos podem então ser catalogados de acordo com a natureza de
seu elemento dinâmico \cite{psaraftis_dynamic_2015}.
Sendo que esta pode ser manifestada em diversas formas, como por exemplo,
dinamismo no conjunto de pedidos, sendo que pedidos podem surgir ou ser
cancelados a qualquer instante.
Outro exemplo seria dinamismo com relação aos tempos de viagem, que podem
variar no decorrer do dia, devido a horários de trânsito intenso ou de
acidentes veiculares.

Apesar da natureza do elemento dinâmico de um problema poder surgir de qualquer
tipo de evento dinâmico possível em uma operação de transporte, como quebra de
veículos, perda de carga, acidentes de diversos tipos, entre outros, dos 117
artigos estudados e catalogados por  \textcite{psaraftis_dynamic_2015}, 80\%
deles envolvem o surgimento de novos pedidos.

Estendendo as formulações do DARP e do PDPTW apresentadas nas
Seções~\ref{sec:formulacao_DARP}~e~\ref{sec:formulacao_PDPTW} gera-se uma
formulação dinâmica em que o surgimento de novos pedidos como a natureza de 
seu elemento dinâmico.

Para que pedidos cheguem a qualquer instante contido dentro
do intervalo $[0, \planingHorizon]$ define-se um instante 
$\arrivalTime_\request, \forall \request \in \requests$ representando 
o exato momento que o pedido $\request$ se faz conhecido pelo sistema de 
transporte.
O instante de chegada de um pedido implica que este será levado
em consideração para o planejamento das rotas apenas durante a
operação do sistema de transporte e somente quando o instante atual for
igual ou maior que $\arrivalTime_\request$.
Deve-se garantir que o valor $\arrivalTime_\request$ seja menor ou igual ao 
limite inferior da janela de tempo de um pedido 
($\earliestTimeWindow_\request$).

Com a adição desse conjunto de parâmetros à formulação do DARP e do PDPTW
anteriormente descritas obtemos uma formulação sucinta para o DDARP e o DPDPTW.
Destaca-se que o DARP e do PDPTW podem ser considerados como simplificações do
DDARP e do DPDPTW em que $\arrivalTime_\request = 0, \forall \request \in
\requests$.

