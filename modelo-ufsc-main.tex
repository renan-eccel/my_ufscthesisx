%% modelo-ufsc-main.tex
%%
%% Copyright 2017 Evandro Coan
%% Copyright 2012-2016 by abnTeX2 group at http://www.abntex.net.br/
%%
%% This work may be distributed and/or modified under the
%% conditions of the LaTeX Project Public License, either version 1.3
%% of this license or (at your option) any later version.
%% The latest version of this license is in
%%   http://www.latex-project.org/lppl.txt
%% and version 1.3 or later is part of all distributions of LaTeX
%% version 2005/12/01 or later.
%%
%% This work has the LPPL maintenance status `maintained'.
%% The Current Maintainer of this work is the Evandro Coan.
%%
%% The last Maintainer of this work was the abnTeX2 team, led
%% by Lauro César Araujo. Further information are available on
%% https://www.abntex.net.br/
%%
%% This work consists of a bunch of files. But originally there ware 3 files
%% which are renamed as follows:
%% Renamed the `abntex2-modelo-references.bib` to `modelo-ufsc-references.bib`
%% Renamed the `abntex2-modelo-include-comandos` to `Capitulos\01\CH01.tex`
%% Renamed the `abntex2-modelo-trabalho-academico.tex` to `modelo-ufsc-main.tex`
%%

% ------------------------------------------------------------------------
% ------------------------------------------------------------------------
% abnTeX2: Modelo de Trabalho Academico (tese de doutorado, dissertacao de
% mestrado e trabalhos monograficos em geral) em conformidade com
% ABNT NBR 14724:2011: Informacao e documentacao - Trabalhos academicos -
% Apresentacao
% ------------------------------------------------------------------------
% ------------------------------------------------------------------------

% You need to run `pdfTeX` 5 times on the following order: 1. `pdfTeX`, 2. `bibtex`, 3. `pdfTeX` 4.
% `pdfTeX` 5. `pdfTeX` 6. `pdfTeX`, because the bibliography includes a cyclic reference to another
% bibliography, so we need a last pass to fix the bibliography undefined references:
%
% To fix the warning `LaTeX Warning: Label(s) may have changed. Rerun to get cross-references right`,
% open the file `D:\User\Documents\latex\texmfs\install\tex\latex\abntex2\abntex2cite.sty` and
% comment out these two lines:
% 547: % \renewcommand{\bibcite}[2]{%
% 548: %   \@newl@bel{b}{#1}{\hyper@@link[cite]{}{cite.#1}{#2}}}%
\input{setup/setup}
\input{utilities/utilities}

% Thesis settings
%
% ifpackageloaded question
% https://tex.stackexchange.com/questions/70212/ifpackageloaded-question
\makeatletter
\@ifpackageloaded{setup/ufscthesis}
{
    % Informações de dados para CAPA e FOLHA DE ROSTO
    \titulo{Modelo Canônico de TCC, Monografia,  Dissertação, Tese ou Relatório}
    \subtitulo{Pós--Doutorado da UFSC com \abnTeX{}}
    \autor{Nome do Autor}
    \local{Florianópolis, Santa Catarina -- Brasil}
    %\local{Florianópolis}
    \data{\today}
    \orientador{Prof. Dr. Nome do Orientador}
    \coorientador{Prof. Dr. Nome do Coorientador}
    \instituicao{Universidade Federal de Santa Catarina -- UFSC}
    \tipotrabalho{Tese de Doutorado}
    \programa{Programa de Pós--Graduação em Engenharia Elétrica -- PGEEL}
    \centro{Departamento de Engenharia Elétrica e Eletrônica -- EEL}

    % O preambulo deve conter o tipo do trabalho, o objetivo. O nome da instituição e a área de
    % concentração.
    \preambulo{Tese submetida ao Programa de Pós--Graduação em Engenharia Elétrica da Universidade
    Federal de Santa Catarina para a obtenção do Grau de Doutor em Engenharia Elétrica.}

    % Back references translation
    \ifenglish
        \usepackage[hyperpageref]{backref}
    \else
        \@ifpackageloaded{listings}
        {
            % Listing -> Codigo fonte
            \renewcommand{\lstlistingname}{Código--fonte}

            % List of Listings -> Lista de códigos-fonte
            \renewcommand{\lstlistlistingname}{Lista de códigos--fonte}
        }

        % Configurações do pacote backref, paginas com as citações na bibliografia
        \usepackage[brazilian,hyperpageref]{backref}

        % Usado sem a opção hyperpageref de backref
        \renewcommand{\backrefpagesname}{Citado na(s) página(s):~}

        % Texto padrão antes do número das páginas
        \renewcommand{\backref}{}

        % Define os textos da citação
        \renewcommand*{\backrefalt}[4]{
            \ifcase #1 %
                Nenhuma citação no texto.%
            \or
                Citado na página #2.%
            \else
                Citado #1 vezes nas páginas #2.%
            \fi}%
    \fi

    \@ifpackageloaded{definecolor}
    {
        % RGB colors in absolute values from 0 to 255 by using `RGB` tag
        \definecolor{darkblue}{RGB}{26,13,178}

        % Definição de cores, RGB colors in percentage notation by using `rgb` tag
        \definecolor{mygreen}{rgb}{0,0.6,0}
        \definecolor{mygray}{rgb}{0.5,0.5,0.5}
        \definecolor{mymauve}{rgb}{0.58,0,0.82}

        % Configurações de aparência do PDF final
        \definecolor{figcolor}{rgb}{1,0.4,0}  % orange
        \definecolor{tabcolor}{rgb}{1,0.4,0}  % orange
        \definecolor{eqncolor}{rgb}{1,0.4,0}  % orange
        \definecolor{linkcolor}{rgb}{1,0.4,0} % orange
        \definecolor{citecolor}{rgb}{1,0.4,0} % orange
        \definecolor{seccolor}{rgb}{0,0,1}    % blue
        \definecolor{abscolor}{rgb}{0,0,1}    % blue
        \definecolor{titlecolor}{rgb}{0,0,1}  % blue
        \definecolor{biocolor}{rgb}{0,0,1}    % blue

        % Alterando o aspecto da cor azul
        \definecolor{blue}{RGB}{41,5,195}

        % Informações do PDF
        \hypersetup
        {
            % backref,pagebackref=true, % Package hyperref Warning: Option `pagebackref' has already been used
            pdftitle={\@title},
            pdfauthor={Autores},
            pdfsubject={Resumo do artigo},
            pdfcreator={LaTeX with abnTeX2},
            pdfkeywords={abnt}{latex}{UFSC}{abntex2}{tese},
            colorlinks=true, % false: boxed links; true: colored links
            linkcolor=darkblue, % color of internal links
            citecolor=darkgreen, % color of links to bibliography
            filecolor=black, % color of file links
            urlcolor=linkcolor,
            bookmarksdepth=4
        }
    }{}

    % Espaçamentos entre linhas e parágrafos
    %
    % Estilo de capítulos, ver classe para maiores detalhes.Veja outros estilos em:
    % http://www.tex.ac.uk/tex-archive/info/MemoirChapStyles/MemoirChapStyles.pdf
    %
    % \chapterstyle{default}
    % \chapterstyle{pedersen}
    % \chapterstyle{lyhne}
    % \chapterstyle{madsen}
    % \chapterstyle{veelo}
    % \chapterstyle{companion}
    % \chapterstyle{thatcher}
    % \chapterstyle{verville}
    \chapterstyle{VZ14}
    \setlength\beforechapskip{0pt}
    \setlength\midchapskip{15pt}
    \setlength\afterchapskip{15pt}

    % O tamanho do parágrafo é dado por:
    \setlength{\parindent}{1.3cm}

    % Controle do espaçamento entre um parágrafo e outro. Tente também
    % \onelineskip
    \setlength{\parskip}{0.2cm}
}{}
\makeatother

% When writing a large document, it is sometimes useful to work on selected sections of the document
% to speed up compilation time: https://en.wikibooks.org/wiki/TeX/includeonly
% \includeonly{pretexto/agradecimentos}
% \includeonly{pretexto/epigrafe}
% \includeonly{pretexto/fichacatalografica}
% \includeonly{pretexto/folhadeaprovacao}
% \includeonly{pretexto/resumos}
% \includeonly{pretexto/siglas}
% \includeonly{pretexto/simbolos}

% \includeonly{chapters/chapter_1}
% \includeonly{chapters/chapter_2}
% \includeonly{chapters/chapter_3}
% \includeonly{chapters/chapter_4}
% \includeonly{chapters/chapter_5}
% \includeonly{chapters/chapter_6}
% \includeonly{chapters/conclusion}

% \includeonly{postexto/anexo_a}
% \includeonly{postexto/anexo_b}
% \includeonly{postexto/anexo_c}
% \includeonly{postexto/apendice_a}
% \includeonly{postexto/apendice_b}
% \includeonly{postexto/apendice_c}



\begin{document}

% Retira espaço extra obsoleto entre as frases `Double space between sentences`
% https://tex.stackexchange.com/questions/4705/double-space-between-sentences
\frenchspacing



% How to fix destination with the same identifier (name{page.A}) has been already used, duplicate ignored?
% https://tex.stackexchange.com/questions/386446/how-to-fix-destination-with-the-same-identifier-namepage-a-has-been-already
\hypersetup{pageanchor=false}




% ELEMENTOS PRÉ-TEXTUAIS
\includepdf{Imagens/FrenteCapaUFSC.pdf}%

\pretextual

% Capa
\imprimircapa

% Folha de rosto (o * indica que haverá a ficha bibliográfica)
\imprimirfolhaderosto*

% Inserir a ficha bibliografica
%
% Isto é um exemplo de Ficha Catalográfica, ou ``Dados internacionais de
% catalogação-na-publicação''. Você pode utilizar este modelo como referência.
% Porém, provavelmente a biblioteca da sua universidade lhe fornecerá um PDF
% com a ficha catalográfica definitiva após a defesa do trabalho. Quando estiver
% com o documento, salve-o como PDF no diretório do seu projeto e substitua todo
% o conteúdo de implementação deste arquivo pelo comando abaixo:
%
% \begin{fichacatalografica}
%    \includepdf{pretexto/ficha_catalografica.pdf}
% \end{fichacatalografica}


% http://portalbu.ufsc.br/ficha
% http://portal.bu.ufsc.br/servicos/ficha-de-identificacao-da-obra/
\begin{fichacatalografica}
    \vspace*{\fill}

    \begin{center}

        Catalogação na fonte pela Biblioteca Universitária da Universidade Federal de Santa Catarina.

        Arquivo compilado às \currenttime h do dia \today.

        \framebox[\textwidth]
        {
            \begin{minipage}{\textwidth}

                \ttfamily
                \imprimirautor

                \hspace{0.5cm} \imprimirtitulo : \imprimirsubtitulo, / \imprimirautor;
                orientador(a), \imprimirorientador;
                co-orientador(a) -- \imprimirlocal, \currenttime, \imprimirdata.

                % Prints how much pages there are on the document and links to the last page
                \hspace{0.5cm} \pageref{LastPage} p.
                \bigskip

                \hspace{0.5cm} \imprimirtipotrabalho -- \imprimirinstituicao,
                \imprimircentro, \imprimirprograma.
                \bigskip

                \hspace{0.5cm} Inclui referências
                \bigskip

                \hspace{0.5cm}
                    1. Uma Palavra-chave ~
                    2. Outra Palavra-chave ~
                    2. Mais Palavras-chave ~
                    I. \imprimirorientador ~
                    II. \imprimircoorientador ~
                    III. \imprimirprograma ~
                    IV. \imprimirtitulo ~
                \bigskip

                \hspace{7.75cm} CDU 02:141:005.7

            \end{minipage}
        }

    \end{center}

\end{fichacatalografica}



% Inserir errata

% Inserir folha de aprovação

% Isto é um exemplo de Folha de aprovação, elemento obrigatório da NBR
% 14724/2011 (seção 4.2.1.3). Você pode utilizar este modelo até a aprovação
% do trabalho. Após isso, substitua todo o conteúdo deste arquivo por uma
% imagem da página assinada pela banca com o comando abaixo:
% \includepdf{folhadeaprovacao_final.pdf}


\addtotextpreliminarycontent{\lang{Approval Sheet}{Folha de Aprovação}}

\begin{folhadeaprovacao}

    \begin{center}
        {\ABNTEXchapterfont\large\imprimirautor}

        \begin{center}
            \ABNTEXchapterfont\bfseries\Large\imprimirtitulo
        \end{center}

        \begin{minipage}{\textwidth}
            \lang
            {
                This thesis was considered appropriate to obtain the \imprimirformacao, in the area
                of \imprimirarea, and approved in its final form by the \imprimirprograma of the
                \imprimirinstituicao.
            }
            {
                Esta Tese foi julgada adequada para obtenção do Título de \imprimirformacao, na área
                de \imprimirarea, e aprovada em sua forma final pelo \imprimirprograma da
                \imprimirinstituicao.
            }
        \end{minipage}%

    \end{center}
    \begin{center}
        Florianópolis, \imprimirdata.
    \end{center}

    \assinatura{\textbf{Prof. Marcelo Lobo Heldwein, \lang{PhD.}{Dr.}} \\
    \lang{Coordinator of the}{Coordenador do} \imprimirprograma}

    \assinatura{\textbf{\imprimirorientador} \\ \imprimirorientadorRotulo \\
    \imprimirinstituicao~--~\imprimirinstituicaosigla}

    \assinatura{\textbf{\imprimircoorientador} \\ \imprimircoorientadorRotulo \\
    \imprimirinstituicao~--~\imprimirinstituicaosigla}

    \newpage
    \begin{flushleft}
        \textbf{\lang{Examination Board:}{Banca Examinadora:}}
    \end{flushleft}

    \assinatura{\textbf{Prof. Arnaldo José Perin, \lang{PhD.}{Dr.}} \\
    \imprimirinstituicao~--~\imprimirinstituicaosigla}

    \assinatura{\textbf{Prof. Denizar Cruz Martins, \lang{PhD.}{Dr.}} \\
    \imprimirinstituicao~--~\imprimirinstituicaosigla}

    \assinatura{\textbf{Prof. Roberto Francisco Coelho, \lang{PhD.}{Dr.}} \\
    \imprimirinstituicao~--~\imprimirinstituicaosigla}

    \assinatura{\textbf{Prof. Samir Ahmad Mussa, \lang{PhD.}{Dr.}} \\
    \imprimirinstituicao~--~\imprimirinstituicaosigla}

    \assinatura{\textbf{Prof. Telles Brunelli Lazzarin, \lang{PhD.}{Dr.}} \\
    \imprimirinstituicao~--~\imprimirinstituicaosigla}

\end{folhadeaprovacao}



% Dedicatória
\begin{dedicatoria}
    \vspace*{\fill}
    \centering
    \noindent
    \textit{ Este trabalho é dedicado às crianças adultas que,\\
        quando pequenas, sonharam em se tornar cientistas.} \vspace*{\fill}
\end{dedicatoria}

% Agradecimentos
\include{pretexto/agradecimentos}

% Epígrafe
\ifthenelse{ \equal{\WriteEpigraphs}{true} }{
\begin{epigrafe}

    \vspace*{\fill}
    \begin{flushright}
        \textit{``Assim como aquele pecado da juventude, este documento te perseguirá pelo resto da vida.''}\\Enio Valmor Kassick
    \end{flushright}
    \begin{flushright}
        \textit{``Estupidez trará mais autoconfiança do que o conhecimento e a bravura juntas.''}\\
        Adriano Ruseler
    \end{flushright}
    \begin{flushright}
        \textit{``Learn from yesterday, live for today, hope for tomorrow. The important thing is not to stop questioning.''}\\ Albert Einstein
    \end{flushright}
    \begin{flushright}
        \textit{``The true sign of intelligence is not knowledge but imagination.''}\\  Albert Einstein
    \end{flushright}
    \begin{flushright}
        \textit{``Peace cannot be kept by force; it can only be achieved by understanding.''}\\ Albert Einstein
    \end{flushright}
    \begin{flushright}
        \textit{``Whoever is careless with the truth in small matters cannot be trusted with important matters.''}\\    Albert Einstein
    \end{flushright}
    \begin{flushright}
        \textit{``Extraordinary claims require extraordinary evidence''}\\
        Carl Sagan
    \end{flushright}
    \begin{flushright}
        \textit{``Catholic, which I was until I reached the age of reason.''}\\
        George Carlin
    \end{flushright}
    \begin{flushright}
        \textit{``We made too many wrong mistakes.''}\\
        Yogi Berra
    \end{flushright}

\end{epigrafe}
}{
    % do something different..
}

% RESUMOS
%
% Ajusta o espaçamento dos parágrafos do resumo
\setlength{\absparsep}{18pt}
\swapcontents
{
    % Changing babel package inside a single chapter
    % https://tex.stackexchange.com/questions/20987/changing-babel-package-inside-a-single-chapter
    %
    % Multiple-language document - babel - selectlanguage vs begin/end{otherlanguage}
    % https://tex.stackexchange.com/questions/36526/multiple-language-document-babel-selectlanguage-vs-begin-endotherlanguage
    \addtotextpreliminarycontent{English's Abstract}
    \begin{otherlanguage*}{english}
    \begin{resumo}[Abstract]

        This is the english abstract.

        \imprimirpalavraschave{Keywords}{\begin{inparaitem}[]\palavraschaveingles\end{inparaitem}}

    \end{resumo}
    \end{otherlanguage*}
}
{
    \addtotextpreliminarycontent{Resumo em Português}
    \begin{otherlanguage*}{brazil}
    \begin{resumo}[Resumo]

        Segundo a \textcite[3.1-3.2]{NBR6028:2003}, o resumo deve ressaltar o
        objetivo, o método, os resultados e as conclusões do documento. A ordem e a extensão
        destes itens dependem do tipo de resumo (informativo ou indicativo) e do
        tratamento que cada item recebe no documento original. O resumo deve ser
        precedido da referência do documento, com exceção do resumo inserido no
        próprio documento. (\ldots) As palavras-chave devem figurar logo abaixo do
        resumo, antecedidas da expressão Palavras-chave:, separadas entre si por
        ponto e finalizadas também por ponto.

        Além disso, na UFSC o texto do resumo deve ser digitado, em um único bloco, sem espaço de parágrafo. O resumo deve
        ser significativo, composto de uma sequência de frases concisas, afirmativas e não de uma
        enumeração de tópicos. Não deve conter citações. Deve usar o verbo na voz passiva. Abaixo do
        resumo, deve-se informar as palavras-chave (palavras ou expressões significativas retiradas do
        texto) ou, termos retirados de thesaurus da área. 

        \imprimirpalavraschave{Palavras-chaves}{\begin{inparaitem}[]\palavraschaveportugues\end{inparaitem}}

    \end{resumo}
    \end{otherlanguage*}
}



% % resumo em francês
% \addtotextpreliminarycontent{Français Résumé}
% \begin{resumo}[Résumé]
%   \begin{otherlanguage*}{french}
%       Il s'agit d'un résumé en français.

%       \imprimirpalavraschave{Mots-clés}{latex. abntex. publication de textes.}
%   \end{otherlanguage*}
% \end{resumo}


% % resumo em espanhol
% \addtotextpreliminarycontent{Español Resumen}
% \begin{resumo}[Resumen]
%   \begin{otherlanguage*}{spanish}
%       Este es el resumen en español.

%       \imprimirpalavraschave{Palabras clave}{latex. abntex. publicación de textos.}
%   \end{otherlanguage*}
% \end{resumo}





% inserir lista de ilustrações
\pdfbookmark[0]{\listfigurename}{lof}
\listoffigures*
\cleardoublepage

% inserir lista de tabelas
\pdfbookmark[0]{\listtablename}{lot}
\listoftables*
\cleardoublepage

% inserir códigos fonte
% ---
\pdfbookmark[0]{\lstlistingname}{lol}
\lstlistoflistings*
\cleardoublepage

% inserir lista de abreviaturas e siglas


\addtothesispreliminarycontent{\chooselang{List of Acronyms}{Lista de Siglas}}

\begin{siglas}
    \item[ABNT] Associação Brasileira de Normas Técnicas
    \item[abnTeX] ABsurdas Normas para TeX
\end{siglas}



% Inserir lista de símbolos


\addtotextpreliminarycontent{\lang{List of Symbols}{Lista de Símbolos}}

% Devam aparecer na mesma ordem de ocorrência no texto.
\begin{simbolos}
    \item[$\arcs$]{Conjunto dos arcos $\in \graph$}
  \item[$\arrivalTime_\request$] Instante de chegada do pedido $\request$
    \item[$\requestLatestArrivalTime$]
      Último instante possível para a chegada do pedido $\request$
    \item[$\pickupNodes$] Subconjunto dos nós, contendo apenas nós de coleta
    \item[$\arcCost{i}{j}$] Custo de viagem entre os nós $i$ e $j$
    \item[$\deliveryNodes$] Subconjunto dos nós, contendo apenas nós de entrega
    \item[$\dynamism$] Dinamismo
    \item[$\earliestTimeWindow_{\originIndex}$]
      Limite inferior da janela de tempo da coleta do pedido $\request$
    \item[$\earliestTimeWindow_{\destinationIndex}$]
      Limite inferior da janela de tempo da entrega do pedido $\request$
    \item[$\graph$] Grafo
    \item[$\planingHorizon$] Limite superior do horizonte de planejamento
    \item[$\request$] Um elemento do conjunto $\requests$
    \item[$\vehiclesSet$] Conjunto de veículos
    \item[$\vehicle$] Um elemento do conjunto $\vehiclesSet$
    \item[$\vehiclesSetSize$] Tamanho do conjunto de veículos
    \item[$\maxRideTime$] Tempo máximo de viagem global
    \item[$\maxRideTime_\request$] Tempo máximo de viagem do pedido 
      $\request$
    \item[$\latestTimeWindow_{\originIndex}$]
      Limite superior da janela de tempo da coleta do pedido $\request$
    \item[$\latestTimeWindow_{\destinationIndex}$]
      Limite superior da janela de tempo da entrega do pedido $\request$
    \item[$\numberOfRequests$] Número de pedidos de uma instância
    \item[$\requests$] Conjunto dos pedidos de transporte 
    \item[$\capacity$] Capacidade global
    \item[$\vehicleCapacity$] Capacidade do veículo $\vehicle$
    \item[$\load$] Carregamento global
    \item[$\requestLoad$] Carregamento do pedido $\request$
    \item[$\originNodeLoad$] Carregamento do pedido no nó $\startNode$
    \item[$\destinationNodeLoad$]
     Carregamento do pedido no nó $\destinationNode$
    \item[$\nodeServiceTime{\originIndex}$]
      Tempo de serviço no nó $\request$
    \item[$\maxRouteTime$] Tempo de rota máximo global
    \item[$\vehicleMaxRouteTime$] Tempo de rota máximo do veículo $\vehicle$
    \item[$\arcTravelTime{i}{j}$] Tempo de viagem entre os nós $i$ e $j$
    \item[$\urgency$] Urgência
    \item[$\nodes$] Conjunto dos nós $\in \graph$
    \item[$\startNode$] Nó inicial
    \item[$\lastNode$] Nó final 
    \item[$\originNode$] O nó de coleta do pedido $\request$
    \item[$\destinationNode$] O nó de entrega do pedido $\request$
    \item[$\timeWindowWidth$] Largura global das janelas de tempo 
    \item[$\staticPercentage$] Percentagem de pedidos estáticos
    \item[$\maneuverTime$] Tempo de manobra
    \item[$\intervalsBetweenArrivals$]
      Lista de intervalos entre instantes de chegada de pedidos 
      consecutivos
    \item[$\intervalBetweenArrivals_\request$]
      Intervalo de tempo entre os pedidos $i + 1$ e $i$
    \item[$\perfectInterval$] Intervalo perfeito entre instantes de chegada
    \item[$\deviationFromPerfectInterval_\request$]
      Diferença entre o $\perfectInterval$ e 
      $\intervalBetweenArrivals_\request$
    \item[$\normalDistribution{a}{b}$]
       Distribuição normal entre os valores $a$ e $b$
    \item[$\uniformDistribution{a}{b}$]
      Distribuição uniforme entre os valores $a$ e $b$
\end{simbolos}


% Inserir o sumario
\pdfbookmark[0]{\contentsname}{toc}
\tableofcontents*
\cleardoublepage




% How to fix destination with the same identifier (name{page.A}) has been already used, duplicate ignored?
% https://tex.stackexchange.com/questions/386446/how-to-fix-destination-with-the-same-identifier-namepage-a-has-been-already
\hypersetup{pageanchor=true}









% ELEMENTOS TEXTUAIS
%
% Configura estilo das páginas.
\textual

\setlength\beforechapskip{50pt}
\setlength\midchapskip{20pt}
\setlength\afterchapskip{20pt}

% Configura estilo das páginas com logos
% \textualINEPUFSC

% Introdução (exemplo de capítulo sem numeração, mas presente no Sumário)
\include{chapters/chapter_1}

% PARTE
% \part{Preparação da pesquisa}

% Capitulo com exemplos de comandos inseridos de arquivo externo
\include{chapters/chapter_2}

% PARTE
% \part{Referenciais teóricos}

% Capitulo de revisão de literatura
% 
% The \phantomsection command is needed to create a link to a place in the document that is not a
% figure, equation, table, section, subsection, chapter, etc.
\cleardoublepage
\phantomsection


% \MakeUppercase in \section and \chapter with hyperref cause trouble
% https://tex.stackexchange.com/questions/199374/makeuppercase-in-section-and-chapter-with-hyperref-cause-trouble
\addcontentsline{toc}{chapter}{\HyperrefUppercase\chooselang{Chapter 3}{Capítulo 3}}


\chapter{Lorem ipsum dolor sit amet}

\begin{flushright}
    \showfont
\end{flushright}

\newpage


% ---
\section{Aliquam vestibulum fringilla lorem}
% ---

Lipsum me [1]

Lipsum me [2-3]


% PARTE
% \part{Resultados}

% Primeiro capitulo de Resultados
% 
% The \phantomsection command is needed to create a link to a place in the document that is not a
% figure, equation, table, section, subsection, chapter, etc.
\cleardoublepage
\phantomsection


% \MakeUppercase in \section and \chapter with hyperref cause trouble
% https://tex.stackexchange.com/questions/199374/makeuppercase-in-section-and-chapter-with-hyperref-cause-trouble
\addcontentsline{toc}{chapter}{\HyperrefUppercase{Capítulo 4}}


\chapter{Lectus lobortis condimentum}
% ---

\begin{flushright}
    \showfont
\end{flushright}

\newpage

% ---
\section{Vestibulum ante ipsum primis in faucibus orci luctus et ultrices
    posuere cubilia Curae}
% ---

\lipsum[21-22]

\showfont


% Segundo capitulo de Resultados
%
% The \phantomsection command is needed to create a link to a place in the document that is not a
% figure, equation, table, section, subsection, chapter, etc.
\cleardoublepage
\phantomsection


% \MakeUppercase in \section and \chapter with hyperref cause trouble
% https://tex.stackexchange.com/questions/199374/makeuppercase-in-section-and-chapter-with-hyperref-cause-trouble
\addcontentsline{toc}{chapter}{\HyperrefUppercase{Capítulo 5}}


\chapter[Nam sed tellus sit amet lectus]{Nam sed tellus sit amet lectus urna ullamcorper tristique interdum
elementum}
% ---

\begin{flushright}
\showfont
\end{flushright}

\newpage



\section[Some encoding tests]{\showfont}
\subsection{\showfont}
\subsubsection{\showfont}
\subsubsubsection{\showfont}


% ---
\section{Pellentesque sit amet pede ac sem eleifend consectetuer}
% ---

Lipsum me [24-26]


% Finaliza a parte no bookmark do PDF
% para que se inicie o bookmark na raiz
% e adiciona espaço de parte no Sumário
\phantompart

% Conclusão (outro exemplo de capítulo sem numeração e presente no sumário)


% The \phantomsection command is needed to create a link to a place in the document that is not a
% figure, equation, table, section, subsection, chapter, etc.
%
% When do I need to invoke \phantomsection?
% https://tex.stackexchange.com/questions/44088/when-do-i-need-to-invoke-phantomsection
\phantomsection

% ---
% Considerações Finais (outro exemplo de capítulo sem numeração e presente no sumário)
% ---
\chapter*[]{\lang{Final Remarks}{Considerações Finais}}
\phantomsection
\addcontentsline{toc}{chapter}{Considerações Finais}%

    Lipsum me [31-33]












\postextual
\setlength\beforechapskip{0pt}
\setlength\midchapskip{15pt}
\setlength\afterchapskip{15pt}

% Referências bibliográficas
\bibliography{modelo-ufsc-references}








% ELEMENTOS PÓS-TEXTUAIS
\postextual
\setlength\beforechapskip{0pt}
\setlength\midchapskip{15pt}
\setlength\afterchapskip{15pt}

% Referências bibliográficas
\bibliography{modelo-ufsc-references}

% Glossário, consulte o manual da classe abntex2 para orientações sobre o glossário.
% \glossary

% Apêndices, inicia os apêndices
\begin{apendicesenv}

    % Imprime uma página indicando o início dos apêndices
    \partapendices

    \setlength\beforechapskip{50pt}
    \setlength\midchapskip{20pt}
    \setlength\afterchapskip{20pt}

    \include{Apendices/A/APA}
    % \include{Apendices/B/APB}
    % \include{Apendices/C/APC}

\end{apendicesenv}

% Anexos, inicia os anexos
\begin{anexosenv}

    % Imprime uma página indicando o início dos anexos
    \partanexos

    \setlength\beforechapskip{50pt}
    \setlength\midchapskip{20pt}
    \setlength\afterchapskip{20pt}

    \include{Anexos/A/ANA}
    % \include{Anexos/B/ANB}
    % \include{Anexos/C/ANC}

\end{anexosenv}

% INDICE REMISSIVO
\phantompart
\printindex






\end{document}

